% Options for packages loaded elsewhere
\PassOptionsToPackage{unicode}{hyperref}
\PassOptionsToPackage{hyphens}{url}
%
\documentclass[
]{article}
\title{ForeCasting Methods Project}
\author{Group}
\date{01/05/2022}

\usepackage{amsmath,amssymb}
\usepackage{lmodern}
\usepackage{iftex}
\ifPDFTeX
  \usepackage[T1]{fontenc}
  \usepackage[utf8]{inputenc}
  \usepackage{textcomp} % provide euro and other symbols
\else % if luatex or xetex
  \usepackage{unicode-math}
  \defaultfontfeatures{Scale=MatchLowercase}
  \defaultfontfeatures[\rmfamily]{Ligatures=TeX,Scale=1}
\fi
% Use upquote if available, for straight quotes in verbatim environments
\IfFileExists{upquote.sty}{\usepackage{upquote}}{}
\IfFileExists{microtype.sty}{% use microtype if available
  \usepackage[]{microtype}
  \UseMicrotypeSet[protrusion]{basicmath} % disable protrusion for tt fonts
}{}
\makeatletter
\@ifundefined{KOMAClassName}{% if non-KOMA class
  \IfFileExists{parskip.sty}{%
    \usepackage{parskip}
  }{% else
    \setlength{\parindent}{0pt}
    \setlength{\parskip}{6pt plus 2pt minus 1pt}}
}{% if KOMA class
  \KOMAoptions{parskip=half}}
\makeatother
\usepackage{xcolor}
\IfFileExists{xurl.sty}{\usepackage{xurl}}{} % add URL line breaks if available
\IfFileExists{bookmark.sty}{\usepackage{bookmark}}{\usepackage{hyperref}}
\hypersetup{
  pdftitle={ForeCasting Methods Project},
  pdfauthor={Group},
  hidelinks,
  pdfcreator={LaTeX via pandoc}}
\urlstyle{same} % disable monospaced font for URLs
\usepackage[margin=1in]{geometry}
\usepackage{color}
\usepackage{fancyvrb}
\newcommand{\VerbBar}{|}
\newcommand{\VERB}{\Verb[commandchars=\\\{\}]}
\DefineVerbatimEnvironment{Highlighting}{Verbatim}{commandchars=\\\{\}}
% Add ',fontsize=\small' for more characters per line
\usepackage{framed}
\definecolor{shadecolor}{RGB}{248,248,248}
\newenvironment{Shaded}{\begin{snugshade}}{\end{snugshade}}
\newcommand{\AlertTok}[1]{\textcolor[rgb]{0.94,0.16,0.16}{#1}}
\newcommand{\AnnotationTok}[1]{\textcolor[rgb]{0.56,0.35,0.01}{\textbf{\textit{#1}}}}
\newcommand{\AttributeTok}[1]{\textcolor[rgb]{0.77,0.63,0.00}{#1}}
\newcommand{\BaseNTok}[1]{\textcolor[rgb]{0.00,0.00,0.81}{#1}}
\newcommand{\BuiltInTok}[1]{#1}
\newcommand{\CharTok}[1]{\textcolor[rgb]{0.31,0.60,0.02}{#1}}
\newcommand{\CommentTok}[1]{\textcolor[rgb]{0.56,0.35,0.01}{\textit{#1}}}
\newcommand{\CommentVarTok}[1]{\textcolor[rgb]{0.56,0.35,0.01}{\textbf{\textit{#1}}}}
\newcommand{\ConstantTok}[1]{\textcolor[rgb]{0.00,0.00,0.00}{#1}}
\newcommand{\ControlFlowTok}[1]{\textcolor[rgb]{0.13,0.29,0.53}{\textbf{#1}}}
\newcommand{\DataTypeTok}[1]{\textcolor[rgb]{0.13,0.29,0.53}{#1}}
\newcommand{\DecValTok}[1]{\textcolor[rgb]{0.00,0.00,0.81}{#1}}
\newcommand{\DocumentationTok}[1]{\textcolor[rgb]{0.56,0.35,0.01}{\textbf{\textit{#1}}}}
\newcommand{\ErrorTok}[1]{\textcolor[rgb]{0.64,0.00,0.00}{\textbf{#1}}}
\newcommand{\ExtensionTok}[1]{#1}
\newcommand{\FloatTok}[1]{\textcolor[rgb]{0.00,0.00,0.81}{#1}}
\newcommand{\FunctionTok}[1]{\textcolor[rgb]{0.00,0.00,0.00}{#1}}
\newcommand{\ImportTok}[1]{#1}
\newcommand{\InformationTok}[1]{\textcolor[rgb]{0.56,0.35,0.01}{\textbf{\textit{#1}}}}
\newcommand{\KeywordTok}[1]{\textcolor[rgb]{0.13,0.29,0.53}{\textbf{#1}}}
\newcommand{\NormalTok}[1]{#1}
\newcommand{\OperatorTok}[1]{\textcolor[rgb]{0.81,0.36,0.00}{\textbf{#1}}}
\newcommand{\OtherTok}[1]{\textcolor[rgb]{0.56,0.35,0.01}{#1}}
\newcommand{\PreprocessorTok}[1]{\textcolor[rgb]{0.56,0.35,0.01}{\textit{#1}}}
\newcommand{\RegionMarkerTok}[1]{#1}
\newcommand{\SpecialCharTok}[1]{\textcolor[rgb]{0.00,0.00,0.00}{#1}}
\newcommand{\SpecialStringTok}[1]{\textcolor[rgb]{0.31,0.60,0.02}{#1}}
\newcommand{\StringTok}[1]{\textcolor[rgb]{0.31,0.60,0.02}{#1}}
\newcommand{\VariableTok}[1]{\textcolor[rgb]{0.00,0.00,0.00}{#1}}
\newcommand{\VerbatimStringTok}[1]{\textcolor[rgb]{0.31,0.60,0.02}{#1}}
\newcommand{\WarningTok}[1]{\textcolor[rgb]{0.56,0.35,0.01}{\textbf{\textit{#1}}}}
\usepackage{graphicx}
\makeatletter
\def\maxwidth{\ifdim\Gin@nat@width>\linewidth\linewidth\else\Gin@nat@width\fi}
\def\maxheight{\ifdim\Gin@nat@height>\textheight\textheight\else\Gin@nat@height\fi}
\makeatother
% Scale images if necessary, so that they will not overflow the page
% margins by default, and it is still possible to overwrite the defaults
% using explicit options in \includegraphics[width, height, ...]{}
\setkeys{Gin}{width=\maxwidth,height=\maxheight,keepaspectratio}
% Set default figure placement to htbp
\makeatletter
\def\fps@figure{htbp}
\makeatother
\setlength{\emergencystretch}{3em} % prevent overfull lines
\providecommand{\tightlist}{%
  \setlength{\itemsep}{0pt}\setlength{\parskip}{0pt}}
\setcounter{secnumdepth}{-\maxdimen} % remove section numbering
\ifLuaTeX
  \usepackage{selnolig}  % disable illegal ligatures
\fi

\begin{document}
\maketitle

\hypertarget{time-series-graphics}{%
\subsection{2 Time series graphics}\label{time-series-graphics}}

In this section the group will explore the dataset by ploting time
series graphics and analyzing them, this section is really important as
it provides the group an overall view of the temperature changing
through the time and allows the group to identify possible patterns,
such as trends, cycles and seasonalities during the time.

\hypertarget{general-overview-of-the-dataset}{%
\subsubsection{General overview of the
dataset}\label{general-overview-of-the-dataset}}

\begin{Shaded}
\begin{Highlighting}[]
\NormalTok{GlobalTemp }\SpecialCharTok{\%\textgreater{}\%} \FunctionTok{autoplot}\NormalTok{(LandOceanAvgTemp)}\SpecialCharTok{+}
  \FunctionTok{geom\_line}\NormalTok{(}\AttributeTok{color=}\StringTok{"blue"}\NormalTok{)}\SpecialCharTok{+}
  \FunctionTok{labs}\NormalTok{(}\AttributeTok{title =} \StringTok{"Land \& Ocean Temperature {-} Global"}\NormalTok{,}
       \AttributeTok{subtitle =} \StringTok{"Average"}\NormalTok{,}
       \AttributeTok{y=}\StringTok{"LandOceanAvgTemp Cº"}\NormalTok{,}
       \AttributeTok{caption =} \StringTok{"fig. 1"}\NormalTok{)}
\end{Highlighting}
\end{Shaded}

\includegraphics{Project_files/figure-latex/unnamed-chunk-3-1.pdf}

\begin{Shaded}
\begin{Highlighting}[]
\NormalTok{GlobalCountryTemp }\SpecialCharTok{\%\textgreater{}\%} \FunctionTok{filter}\NormalTok{(Country }\SpecialCharTok{==} \StringTok{"Brazil"}\NormalTok{) }\SpecialCharTok{\%\textgreater{}\%} \FunctionTok{autoplot}\NormalTok{(LandAvgTemp)}\SpecialCharTok{+}
  \FunctionTok{geom\_line}\NormalTok{(}\AttributeTok{color=}\StringTok{"green"}\NormalTok{)}\SpecialCharTok{+}
  \FunctionTok{labs}\NormalTok{(}\AttributeTok{title =} \StringTok{"Land Temperature in Brazil {-} Since 1850"}\NormalTok{,}
       \AttributeTok{subtitle=}\StringTok{"Average"}\NormalTok{,}
       \AttributeTok{y=}\StringTok{"LandAvgTemp Cº"}\NormalTok{,}
       \AttributeTok{caption =} \StringTok{"fig. 2"}\NormalTok{)}
\end{Highlighting}
\end{Shaded}

\includegraphics{Project_files/figure-latex/unnamed-chunk-3-2.pdf}

\begin{Shaded}
\begin{Highlighting}[]
\NormalTok{GlobalCountryTemp }\SpecialCharTok{\%\textgreater{}\%} \FunctionTok{filter}\NormalTok{(Country}\SpecialCharTok{==}\StringTok{"Global"}\NormalTok{) }\SpecialCharTok{\%\textgreater{}\%} \FunctionTok{autoplot}\NormalTok{(LandAvgTempUnc)}\SpecialCharTok{+}
  \FunctionTok{labs}\NormalTok{(}\AttributeTok{title =} \StringTok{"Land Average temperature Uncertainty {-} Evolution through time"}\NormalTok{,}
       \AttributeTok{caption =} \StringTok{"fig. 3"}\NormalTok{)}
\end{Highlighting}
\end{Shaded}

\includegraphics{Project_files/figure-latex/unnamed-chunk-3-3.pdf}

The dataset provides three different views that can be combined and
analysed in multiple different ways.

The first view is the Global Land \& Ocean average temperature (see
fig.~1), it provides an interesting sight on how the combination of both
land and ocean temperature has been changing since 1850.

The central view of the dataset is the Land temperature analysis, which
can be divided into two parts:

\begin{itemize}
\tightlist
\item
  A global analysis that contains average, max and min temperature
  values
\item
  A more narrow analysis by country that allows us to observe particular
  countries average temperatures over the years. Note that the years in
  analyse may differ from country to country due to missing data in the
  dataset, especially for years before 1900. Figure 2 shows the time
  series of Land Temperature of Brazil.
\end{itemize}

The final view is related with the uncertainty of the average
temperature values and how it has changed over time (see fig.~3).

\begin{Shaded}
\begin{Highlighting}[]
\NormalTok{GlobalTempV }\SpecialCharTok{\%\textgreater{}\%} \FunctionTok{ggplot}\NormalTok{()}\SpecialCharTok{+}
  \FunctionTok{geom\_line}\NormalTok{(}\AttributeTok{data=}\NormalTok{GlobalTempV, }\FunctionTok{aes}\NormalTok{(Date, LandMaxTemp, }\AttributeTok{col=}\StringTok{"LandMaxTemp"}\NormalTok{))}\SpecialCharTok{+}
  \FunctionTok{geom\_line}\NormalTok{(}\AttributeTok{data=}\NormalTok{GlobalTempV, }\FunctionTok{aes}\NormalTok{(Date,LandAvgTemp,}\AttributeTok{col=}\StringTok{"LandAvgTemp"}\NormalTok{),}\AttributeTok{lwd=}\DecValTok{2}\NormalTok{)}\SpecialCharTok{+}
  \FunctionTok{geom\_line}\NormalTok{(}\AttributeTok{data=}\NormalTok{GlobalTempV, }\FunctionTok{aes}\NormalTok{(Date, LandMinTemp, }\AttributeTok{col=}\StringTok{"LandMinTemp"}\NormalTok{))}\SpecialCharTok{+}
  \FunctionTok{labs}\NormalTok{(}\AttributeTok{title =} \StringTok{"Land temperature {-} Global overview"}\NormalTok{,}
       \AttributeTok{y=}\StringTok{"LandTemp Cº"}\NormalTok{,}
       \AttributeTok{color=}\StringTok{"Temperature Type"}\NormalTok{,}
       \AttributeTok{caption =} \StringTok{"fig. 4"}\NormalTok{)}\SpecialCharTok{+}
  \FunctionTok{scale\_color\_manual}\NormalTok{(}\AttributeTok{values=} \FunctionTok{c}\NormalTok{(}\StringTok{"red"}\NormalTok{,}\StringTok{"black"}\NormalTok{,}\StringTok{"blue"}\NormalTok{))}
\end{Highlighting}
\end{Shaded}

\includegraphics{Project_files/figure-latex/unnamed-chunk-4-1.pdf}

\begin{Shaded}
\begin{Highlighting}[]
\NormalTok{GlobalCountryTemp }\SpecialCharTok{\%\textgreater{}\%} \FunctionTok{filter}\NormalTok{(Country }\SpecialCharTok{\%in\%}\NormalTok{ countries) }\SpecialCharTok{\%\textgreater{}\%} \FunctionTok{filter}\NormalTok{(}\FunctionTok{year}\NormalTok{(Date)}\SpecialCharTok{\textgreater{}}\DecValTok{1984}\NormalTok{) }\SpecialCharTok{\%\textgreater{}\%} \FunctionTok{autoplot}\NormalTok{(LandAvgTemp)}\SpecialCharTok{+}
  \FunctionTok{geom\_line}\NormalTok{(}\AttributeTok{data=}\NormalTok{GlobalTempV ,}\FunctionTok{aes}\NormalTok{(Date, LandAvgTemp, }\AttributeTok{col=} \StringTok{"Global"}\NormalTok{),}\AttributeTok{lwd=}\DecValTok{2}\NormalTok{)}\SpecialCharTok{+}
  \FunctionTok{labs}\NormalTok{(}\AttributeTok{title=} \StringTok{"Global \& Country Land Temperature {-} Last 30 years"}\NormalTok{,}
            \AttributeTok{subtitle =} \StringTok{"Average"}\NormalTok{,}
            \AttributeTok{y=} \StringTok{"LandAvgTemp Cº"}\NormalTok{,}
       \AttributeTok{caption =} \StringTok{"fig. 5"}\NormalTok{)}\SpecialCharTok{+}
  \FunctionTok{ylim}\NormalTok{(}\SpecialCharTok{{-}}\DecValTok{30}\NormalTok{,}\DecValTok{40}\NormalTok{)}\SpecialCharTok{+}
  \FunctionTok{theme}\NormalTok{(}\AttributeTok{legend.position =} \StringTok{"right"}\NormalTok{)}
\end{Highlighting}
\end{Shaded}

\includegraphics{Project_files/figure-latex/unnamed-chunk-4-2.pdf} !
@Rodrigo: You should also add a small description for fig.~4. What can
we see?

Figure 5 shows the average temperature in the last 30 years of three
selected countries and ``global''. As we start looking at these time
series plots we can see that the temperature behaves differently
according to the countries analysed, some countries belong to hot
regions like Mali and even Portugal and therefore have higher
temperatures while other countries like Greenland are in more northern
zones and have colder temperatures. The global temperature is nothing
else but the weighted average of the temperature of all countries in the
original dataset and therefore it appears on the ``middle'' of the time
series graph. The same reasoning can be applied to the max and min
global temperature in figure 4.

When we take a closer look to these graphs it's possible to start
observing some seasonal and trending patterns. Regarding the trending
patterns, we can clearly see a very smooth crescent trend when it comes
to land and ocean temperatures (fig.~1), which makes sense as a result
of a direct effect of the global warming in the temperature, making it
rise. On the other hand, there is a very clear decrescent trending on
the temperature uncertainty (fig.~3), which makes sense from a point
where as the time pass, the practices to measure temperature are better
and more precise.

Looking now at the seasonal pattern, when we highlight the two graphs
starting from 1985 onwards (fig.~4 and 5), it becomes explicit that
there is a seasonal behavior every year. There are four seasons (winter,
spring, summer and autumn) and we can clearly see that each of those has
its temperature characteristics.

\hypertarget{focusing-on-seasonality-cycles-and-trends}{%
\subsubsection{Focusing on seasonality, cycles and
trends}\label{focusing-on-seasonality-cycles-and-trends}}

In this section, the objective is to focus on the seasonal and trending
pattern identified previously and confirm them.

\begin{Shaded}
\begin{Highlighting}[]
\CommentTok{\#seasonal and trending}
\NormalTok{GlobalCountryTemp }\SpecialCharTok{\%\textgreater{}\%} 
  \FunctionTok{filter}\NormalTok{(Country}\SpecialCharTok{==}\StringTok{"Global"}\NormalTok{) }\SpecialCharTok{\%\textgreater{}\%} 
  \FunctionTok{gg\_season}\NormalTok{(LandAvgTemp)}\SpecialCharTok{+}
  \FunctionTok{labs}\NormalTok{(}\AttributeTok{title =} \StringTok{"Seasonality in global temperature {-} Land"}\NormalTok{,}
       \AttributeTok{y=}\StringTok{"LandAvgTemp Cº"}\NormalTok{,}
       \AttributeTok{color=}\StringTok{"Years"}\NormalTok{,}
       \AttributeTok{caption =} \StringTok{"fig. 6"}\NormalTok{)}
\end{Highlighting}
\end{Shaded}

\includegraphics{Project_files/figure-latex/unnamed-chunk-5-1.pdf}

\begin{Shaded}
\begin{Highlighting}[]
\NormalTok{GlobalCountryTemp }\SpecialCharTok{\%\textgreater{}\%} 
  \FunctionTok{filter}\NormalTok{(Country}\SpecialCharTok{==}\StringTok{"Global"}\NormalTok{) }\SpecialCharTok{\%\textgreater{}\%} 
  \FunctionTok{gg\_season}\NormalTok{(LandOceanAvgTemp)}\SpecialCharTok{+}
  \FunctionTok{labs}\NormalTok{(}\AttributeTok{title =} \StringTok{"Seasonality in global temperature {-} Land and Ocean"}\NormalTok{,}
       \AttributeTok{y=}\StringTok{"LandOceanAvgTemp Cº"}\NormalTok{,}
       \AttributeTok{color=}\StringTok{"Years"}\NormalTok{,}
       \AttributeTok{caption =} \StringTok{"fig. 7"}\NormalTok{)}
\end{Highlighting}
\end{Shaded}

\includegraphics{Project_files/figure-latex/unnamed-chunk-5-2.pdf}

These graphs (figure 5 and 6) give us some very interesting insights on
the previous spotted seasonality.

We can clearly see that in winter months (December, January and
February), the temperature reaches the lowest values while in summer
months (June, July, August), the temperature reaches its peak. Besides
showing the seasonality, this graph clearly shows the crescent trending
as the years go by. The lines that represent the years keep going up in
the graph.

Another very important aspect that we can spot is that in years around
1800 the values have some kind of spikes, this results from the fact
that there was a big uncertainty around the temperature in these years
and as time pass by, the uncertainty decreases and the lines become more
stable. The land and ocean seasonality graph clearly shows that
difference as it doesn't have values below 1900.

\begin{Shaded}
\begin{Highlighting}[]
\CommentTok{\#subseries}
\NormalTok{GlobalCountryTemp }\SpecialCharTok{\%\textgreater{}\%} 
  \FunctionTok{filter}\NormalTok{(Country}\SpecialCharTok{==}\StringTok{"Global"}\NormalTok{) }\SpecialCharTok{\%\textgreater{}\%} 
  \FunctionTok{gg\_subseries}\NormalTok{(LandAvgTemp)}\SpecialCharTok{+}
  \FunctionTok{labs}\NormalTok{(}\AttributeTok{title =} \StringTok{"Subseries {-} monthly global analysis"}\NormalTok{,}
       \AttributeTok{y=}\StringTok{"LandAvgTemp Cº"}\NormalTok{,}
       \AttributeTok{caption =} \StringTok{"fig. 8"}\NormalTok{)}
\end{Highlighting}
\end{Shaded}

\includegraphics{Project_files/figure-latex/unnamed-chunk-6-1.pdf}

\begin{Shaded}
\begin{Highlighting}[]
\NormalTok{GlobalCountryTemp }\SpecialCharTok{\%\textgreater{}\%} 
  \FunctionTok{filter}\NormalTok{(Country }\SpecialCharTok{\%in\%}\NormalTok{ countries2) }\SpecialCharTok{\%\textgreater{}\%} 
  \FunctionTok{gg\_subseries}\NormalTok{(LandAvgTemp)}\SpecialCharTok{+}
  \FunctionTok{labs}\NormalTok{(}\AttributeTok{title =} \StringTok{"Subseries {-} monthly analysis"}\NormalTok{,}
         \AttributeTok{y=}\StringTok{"LandAvgTemp Cº"}\NormalTok{,}
       \AttributeTok{caption =} \StringTok{"fig. 9"}\NormalTok{)}
\end{Highlighting}
\end{Shaded}

\includegraphics{Project_files/figure-latex/unnamed-chunk-6-2.pdf}

\begin{Shaded}
\begin{Highlighting}[]
\NormalTok{GlobalCountryTemp }\SpecialCharTok{\%\textgreater{}\%} 
  \FunctionTok{filter}\NormalTok{(Country }\SpecialCharTok{\%in\%}\NormalTok{ countries3) }\SpecialCharTok{\%\textgreater{}\%} 
  \FunctionTok{gg\_subseries}\NormalTok{(LandAvgTemp)}\SpecialCharTok{+}
  \FunctionTok{labs}\NormalTok{(}\AttributeTok{title =} \StringTok{"Subseries {-} monthly analysis"}\NormalTok{,}
         \AttributeTok{y=}\StringTok{"LandAvgTemp Cº"}\NormalTok{,}
       \AttributeTok{caption =} \StringTok{"fig. 10"}\NormalTok{)}
\end{Highlighting}
\end{Shaded}

\includegraphics{Project_files/figure-latex/unnamed-chunk-6-3.pdf}

The subseries analysis also provides some more details regarding the
seasonality and trending.

Especially figure 8 shows clearly high discrepancies in early 1800
temperature values due to the high uncertainty. However, it is still
clear to see that there is a growing trend, at a global perspective.

The country subseries analysis in figure 9 and 10 reveals again a strong
seasonality. However, it also shows a new perspective regarding the
seasonality. Some countries have their seasonality ``swaped'' when
compared to the global seasonality. This is due to the fact that the
temperature in the countries of the northern hemisphere is the opposite
of that in the countries of the southern hemisphere: If it's summer in
the north (June, July, August), it's winter in the south. Nevertheless
we can still identify the growing trend of the temperature in most
countries.

\hypertarget{looking-at-autocorrelation-and-white-noise}{%
\subsubsection{Looking at autocorrelation and white
noise}\label{looking-at-autocorrelation-and-white-noise}}

Lastly, we analyse the autocorrelation and examine whether white noise
exists.

\begin{Shaded}
\begin{Highlighting}[]
\CommentTok{\#6, 12 or 24 lags{-}{-}\textgreater{} monthly data}
\CommentTok{\#In this case 12 looks the best as it captures all months of one year}

\NormalTok{GlobalCountryTemp }\SpecialCharTok{\%\textgreater{}\%} \FunctionTok{filter}\NormalTok{(Country}\SpecialCharTok{==}\StringTok{"Global"}\NormalTok{) }\SpecialCharTok{\%\textgreater{}\%} \FunctionTok{ACF}\NormalTok{(LandAvgTemp,}\AttributeTok{lag\_max =} \DecValTok{12}\NormalTok{) }\SpecialCharTok{\%\textgreater{}\%} \FunctionTok{autoplot}\NormalTok{()}\SpecialCharTok{+}
  \FunctionTok{labs}\NormalTok{(}\AttributeTok{title =} \StringTok{"Autocorrelation in LandAvgTemp {-} Global"}\NormalTok{,}
       \AttributeTok{subtitle =} \StringTok{"12 month lag"}\NormalTok{,}
       \AttributeTok{caption =} \StringTok{"fig. 11"}\NormalTok{)}
\end{Highlighting}
\end{Shaded}

\includegraphics{Project_files/figure-latex/unnamed-chunk-7-1.pdf}

\begin{Shaded}
\begin{Highlighting}[]
\NormalTok{GlobalCountryTemp }\SpecialCharTok{\%\textgreater{}\%} 
  \FunctionTok{filter}\NormalTok{(Country }\SpecialCharTok{\%in\%}\NormalTok{ countries2)}\SpecialCharTok{\%\textgreater{}\%} \FunctionTok{ACF}\NormalTok{(LandAvgTemp,}\AttributeTok{lag\_max =} \DecValTok{12}\NormalTok{) }\SpecialCharTok{\%\textgreater{}\%} \FunctionTok{autoplot}\NormalTok{()}\SpecialCharTok{+}
  \FunctionTok{labs}\NormalTok{(}\AttributeTok{title =} \StringTok{"Autocorrelation in LandAvgTemp {-} Country"}\NormalTok{,}
       \AttributeTok{subtitle =} \StringTok{"12 month lag"}\NormalTok{,}
       \AttributeTok{caption =} \StringTok{"fig. 12"}\NormalTok{)}
\end{Highlighting}
\end{Shaded}

\includegraphics{Project_files/figure-latex/unnamed-chunk-7-2.pdf}

\begin{Shaded}
\begin{Highlighting}[]
\CommentTok{\#transforming it into yearly data since there is no seasonality}
\CommentTok{\#showing the first 48 years}
\NormalTok{uncertainty}\OtherTok{\textless{}{-}}\NormalTok{ GlobalCountryTemp }\SpecialCharTok{\%\textgreater{}\%} \FunctionTok{filter}\NormalTok{(Country}\SpecialCharTok{==}\StringTok{"Global"}\NormalTok{) }\SpecialCharTok{\%\textgreater{}\%}
  \FunctionTok{mutate}\NormalTok{(}\AttributeTok{Year=} \FunctionTok{year}\NormalTok{(Date)) }\SpecialCharTok{\%\textgreater{}\%} 
  \FunctionTok{as\_tibble}\NormalTok{(GlobalCountryTemp) }\SpecialCharTok{\%\textgreater{}\%} 
  \FunctionTok{group\_by}\NormalTok{(Year)}\SpecialCharTok{\%\textgreater{}\%}
  \FunctionTok{select}\NormalTok{(}\SpecialCharTok{{-}}\NormalTok{Date) }\SpecialCharTok{\%\textgreater{}\%}
  \FunctionTok{summarise}\NormalTok{(}\AttributeTok{LandAvgTempUnc=} \FunctionTok{mean}\NormalTok{(LandAvgTempUnc))}

\NormalTok{uncertainty }\SpecialCharTok{\%\textgreater{}\%}\FunctionTok{as\_tsibble}\NormalTok{(}\AttributeTok{index=}\NormalTok{Year) }\SpecialCharTok{\%\textgreater{}\%} \FunctionTok{filter}\NormalTok{(Year}\SpecialCharTok{\textgreater{}} \StringTok{"1752"}\NormalTok{) }\SpecialCharTok{\%\textgreater{}\%} 
  \FunctionTok{ACF}\NormalTok{(LandAvgTempUnc, }\AttributeTok{lag\_max =} \DecValTok{48}\NormalTok{) }\SpecialCharTok{\%\textgreater{}\%} \FunctionTok{autoplot}\NormalTok{()}\SpecialCharTok{+}
  \FunctionTok{labs}\NormalTok{(}\AttributeTok{title =} \StringTok{"Autocorrelation in LandAvgTempUnc {-} Global"}\NormalTok{,}
       \AttributeTok{subtitle =} \StringTok{"48 years lag"}\NormalTok{,}
       \AttributeTok{caption =} \StringTok{"fig. 13"}\NormalTok{)}
\end{Highlighting}
\end{Shaded}

\includegraphics{Project_files/figure-latex/unnamed-chunk-7-3.pdf}

By looking at the autocorrelation plots, regarding the Land Average
Temperature (see figure 11 and 12), we used a lag of 12 months since
it's more appropriate for monthly data and it captures all the months of
one year. It clearly shows that autocorrelation exists between the time
series and its lags period. We can also spot that the autocorrelation
peaks on month 6 and 12 which correspond to the highest and lowest
temperature peeks of the seasonality, respectively.

For the autocorrelation analysis of Land Average temperature Uncertainty
(see figure 13), we first changed the monthly data to yearly data as
there is no seasonality and it helps on the visualization. Moreover, we
defined a lag period of 48 years which is about 1/3 of our total data.
We can observe that it follows the typical trend behavior: Small lags
have larger values as they are close to the previous value and tend to
get smaller as the time goes by.

Since our autocorrelation values are mostly above the significant
interval our time series don't follow ``white noise''.

\hypertarget{time-series-decomposition}{%
\section{3. Time Series Decomposition}\label{time-series-decomposition}}

First of all, we are replacing missing values, by the last available
value for that variable (in a up-down manner), so that we keep our time
series without missing gaps, to be able to decompose it.

\begin{Shaded}
\begin{Highlighting}[]
\NormalTok{GlobalCountryTemp }\SpecialCharTok{\%\textgreater{}\%}
  \FunctionTok{fill}\NormalTok{(LandAvgTemp, }\AttributeTok{.direction =} \StringTok{\textquotesingle{}down\textquotesingle{}}\NormalTok{) }\SpecialCharTok{\%\textgreater{}\%}
  \FunctionTok{fill}\NormalTok{(LandOceanAvgTemp, }\AttributeTok{.direction =} \StringTok{\textquotesingle{}down\textquotesingle{}}\NormalTok{) }\OtherTok{{-}\textgreater{}}\NormalTok{ GlobalCountryTemp }
  
\CommentTok{\#colSums(is.na(GlobalCountryTemp))}
\end{Highlighting}
\end{Shaded}

Now we are going to see if the series need adjustments. As we can see in
the graphs below, our variation does not change with the level of the
time series. So, we will not use mathematical transformations.

\begin{Shaded}
\begin{Highlighting}[]
\NormalTok{GlobalTemp }\SpecialCharTok{\%\textgreater{}\%}
  \FunctionTok{filter}\NormalTok{(}\FunctionTok{year}\NormalTok{(Date)}\SpecialCharTok{\textgreater{}}\DecValTok{1838}\NormalTok{, }\FunctionTok{year}\NormalTok{(Date)}\SpecialCharTok{\textless{}}\DecValTok{1900}\NormalTok{) }\SpecialCharTok{\%\textgreater{}\%}
  \FunctionTok{autoplot}\NormalTok{(LandAvgTemp)}\SpecialCharTok{+}
  \FunctionTok{geom\_line}\NormalTok{(}\AttributeTok{color =} \StringTok{\textquotesingle{}\#0088cc\textquotesingle{}}\NormalTok{)}\SpecialCharTok{+}
  \FunctionTok{labs}\NormalTok{(}\AttributeTok{title =} \StringTok{"Land Temperature{-} Global"}\NormalTok{,}
       \AttributeTok{subtitle =} \StringTok{"Average"}\NormalTok{,}
       \AttributeTok{y=}\StringTok{"LandAvgTemp Cº"}\NormalTok{, }\AttributeTok{caption =} \StringTok{"fig. 14"}\NormalTok{)}
\end{Highlighting}
\end{Shaded}

\includegraphics{Project_files/figure-latex/unnamed-chunk-9-1.pdf}

\begin{Shaded}
\begin{Highlighting}[]
\NormalTok{GlobalTemp }\SpecialCharTok{\%\textgreater{}\%}
  \FunctionTok{filter}\NormalTok{(}\FunctionTok{year}\NormalTok{(Date)}\SpecialCharTok{\textgreater{}}\DecValTok{1900}\NormalTok{, }\FunctionTok{year}\NormalTok{(Date)}\SpecialCharTok{\textless{}}\DecValTok{1990}\NormalTok{) }\SpecialCharTok{\%\textgreater{}\%}
  \FunctionTok{autoplot}\NormalTok{(LandAvgTemp)}\SpecialCharTok{+}
  \FunctionTok{geom\_line}\NormalTok{(}\AttributeTok{color =} \StringTok{\textquotesingle{}\#ff3399\textquotesingle{}}\NormalTok{)}\SpecialCharTok{+}
  \FunctionTok{labs}\NormalTok{(}\AttributeTok{title =} \StringTok{"Land Temperature{-} Global"}\NormalTok{,}
       \AttributeTok{subtitle =} \StringTok{"Average"}\NormalTok{,}
       \AttributeTok{y=}\StringTok{"LandAvgTemp Cº"}\NormalTok{, }\AttributeTok{caption =} \StringTok{"fig. 15"}\NormalTok{)}
\end{Highlighting}
\end{Shaded}

\includegraphics{Project_files/figure-latex/unnamed-chunk-9-2.pdf}

\begin{Shaded}
\begin{Highlighting}[]
\NormalTok{GlobalTemp }\SpecialCharTok{\%\textgreater{}\%}
  \FunctionTok{filter}\NormalTok{(}\FunctionTok{year}\NormalTok{(Date)}\SpecialCharTok{\textgreater{}}\DecValTok{1990}\NormalTok{, }\FunctionTok{year}\NormalTok{(Date)}\SpecialCharTok{\textless{}}\DecValTok{2014}\NormalTok{) }\SpecialCharTok{\%\textgreater{}\%}
  \FunctionTok{autoplot}\NormalTok{(LandAvgTemp)}\SpecialCharTok{+}
  \FunctionTok{geom\_line}\NormalTok{(}\AttributeTok{color =} \StringTok{\textquotesingle{}\#cc33ff\textquotesingle{}}\NormalTok{)}\SpecialCharTok{+}
  \FunctionTok{labs}\NormalTok{(}\AttributeTok{title =} \StringTok{"Land Temperature{-} Global"}\NormalTok{,}
       \AttributeTok{subtitle =} \StringTok{"Average"}\NormalTok{,}
       \AttributeTok{y=}\StringTok{"LandAvgTemp Cº"}\NormalTok{, }\AttributeTok{caption =} \StringTok{"fig. 16"}\NormalTok{)}
\end{Highlighting}
\end{Shaded}

\includegraphics{Project_files/figure-latex/unnamed-chunk-9-3.pdf}

As the variations around the trend are of similar magnitudes over time,
we will use an Additive Decomposition. There is a consistent up and down
wave regarding the seasonality, and the trend is going upward, also in a
consistent way, although slowly over time. There is no exponential
growth.

\#\#Decomposing using Classical Decomposition

\begin{Shaded}
\begin{Highlighting}[]
\NormalTok{GlobalTemp }\SpecialCharTok{\%\textgreater{}\%}
  \FunctionTok{select}\NormalTok{(LandOceanAvgTemp) }\OtherTok{{-}\textgreater{}}\NormalTok{ land\_ocean\_global\_temp}

\NormalTok{land\_ocean\_global\_temp }\SpecialCharTok{\%\textgreater{}\%}
  \FunctionTok{model}\NormalTok{(}\FunctionTok{classical\_decomposition}\NormalTok{(LandOceanAvgTemp, }\AttributeTok{type =} \StringTok{"additive"}\NormalTok{)) }\SpecialCharTok{\%\textgreater{}\%}
  \FunctionTok{components}\NormalTok{() }\OtherTok{{-}\textgreater{}}\NormalTok{ classic\_decmp}

\NormalTok{classic\_decmp }\SpecialCharTok{\%\textgreater{}\%}
  \FunctionTok{autoplot}\NormalTok{() }\SpecialCharTok{+}
  \FunctionTok{labs}\NormalTok{(}\AttributeTok{title =} \StringTok{"Land and Ocean Global Temperature Classical Additive Decomposition"}\NormalTok{, }\AttributeTok{caption =} \StringTok{"fig. 17"}\NormalTok{)}
\end{Highlighting}
\end{Shaded}

\includegraphics{Project_files/figure-latex/unnamed-chunk-10-1.pdf}

\begin{Shaded}
\begin{Highlighting}[]
\NormalTok{classic\_decmp }\SpecialCharTok{\%\textgreater{}\%}
  \FunctionTok{autoplot}\NormalTok{(random, }\AttributeTok{color =} \StringTok{\textquotesingle{}\#ff668c\textquotesingle{}}\NormalTok{) }\SpecialCharTok{+}
  \FunctionTok{labs}\NormalTok{(}\AttributeTok{title =} \StringTok{"Variations not Captured by Classical Decomposition in Land and Ocean"}\NormalTok{, }\AttributeTok{caption =} \StringTok{"fig. 18"}\NormalTok{)}
\end{Highlighting}
\end{Shaded}

\includegraphics{Project_files/figure-latex/unnamed-chunk-10-2.pdf}

How is Classical Decomposition for our time series?

As Classical Decomposition assumes that the seasonal component repeats
from year to year, it fits reasonably in our time series, since
temperatures as always higher in the Summer and lower in the Winter, and
this pattern in kept.

Now, we will try to decompose the series into the Seasonal components,
the Trend component and a Remainder, using STL Decomposition.

\#\#Decomposing using STL

\begin{Shaded}
\begin{Highlighting}[]
\NormalTok{dec }\OtherTok{\textless{}{-}}\NormalTok{ land\_ocean\_global\_temp }\SpecialCharTok{\%\textgreater{}\%} 
  \FunctionTok{model}\NormalTok{(}\AttributeTok{stl =} \FunctionTok{STL}\NormalTok{(LandOceanAvgTemp))}

\CommentTok{\#components(dec)}


\NormalTok{GlobalTemp }\SpecialCharTok{\%\textgreater{}\%}
  \FunctionTok{select}\NormalTok{(LandAvgTemp) }\OtherTok{{-}\textgreater{}}\NormalTok{ land\_global\_temp}

  
\NormalTok{dec2 }\OtherTok{\textless{}{-}}\NormalTok{ land\_global\_temp }\SpecialCharTok{\%\textgreater{}\%} 
  \FunctionTok{model}\NormalTok{(}\AttributeTok{stl =} \FunctionTok{STL}\NormalTok{(LandAvgTemp))}

\CommentTok{\#components(dec2)}
\end{Highlighting}
\end{Shaded}

Isolating the trend and the seasonal adjusted temperatures in a graph.

\begin{Shaded}
\begin{Highlighting}[]
\NormalTok{land\_ocean\_global\_temp }\SpecialCharTok{\%\textgreater{}\%}
  \FunctionTok{autoplot}\NormalTok{(LandOceanAvgTemp, }\AttributeTok{color =} \StringTok{\textquotesingle{}gray\textquotesingle{}}\NormalTok{) }\SpecialCharTok{+} 
  \FunctionTok{autolayer}\NormalTok{(}\FunctionTok{components}\NormalTok{(dec), trend, }\AttributeTok{color =} \StringTok{\textquotesingle{}\#6666ff\textquotesingle{}}\NormalTok{) }\SpecialCharTok{+} 
  \FunctionTok{ggtitle}\NormalTok{(}\StringTok{\textquotesingle{}Land and Ocean Global Temperature Trend\textquotesingle{}}\NormalTok{) }\SpecialCharTok{+}
  \FunctionTok{labs}\NormalTok{(}\AttributeTok{caption =} \StringTok{"fig. 19"}\NormalTok{)}
\end{Highlighting}
\end{Shaded}

\includegraphics{Project_files/figure-latex/unnamed-chunk-12-1.pdf}

\begin{Shaded}
\begin{Highlighting}[]
\NormalTok{land\_global\_temp }\SpecialCharTok{\%\textgreater{}\%}
  \FunctionTok{autoplot}\NormalTok{(LandAvgTemp, }\AttributeTok{color =} \StringTok{\textquotesingle{}gray\textquotesingle{}}\NormalTok{) }\SpecialCharTok{+} 
  \FunctionTok{autolayer}\NormalTok{(}\FunctionTok{components}\NormalTok{(dec2), trend, }\AttributeTok{color =} \StringTok{\textquotesingle{}purple\textquotesingle{}}\NormalTok{) }\SpecialCharTok{+} 
  \FunctionTok{ggtitle}\NormalTok{(}\StringTok{\textquotesingle{}Land Global Temperature Trend\textquotesingle{}}\NormalTok{)}\SpecialCharTok{+}
  \FunctionTok{labs}\NormalTok{(}\AttributeTok{caption =} \StringTok{"fig. 20"}\NormalTok{)}
\end{Highlighting}
\end{Shaded}

\includegraphics{Project_files/figure-latex/unnamed-chunk-12-2.pdf}

\begin{Shaded}
\begin{Highlighting}[]
\FunctionTok{components}\NormalTok{(dec) }\SpecialCharTok{\%\textgreater{}\%}
  \FunctionTok{filter}\NormalTok{(}\FunctionTok{year}\NormalTok{(Date)}\SpecialCharTok{\textgreater{}}\DecValTok{1900}\NormalTok{) }\OtherTok{{-}\textgreater{}}\NormalTok{ comp\_after1900}

\NormalTok{land\_ocean\_global\_temp }\SpecialCharTok{\%\textgreater{}\%}
  \FunctionTok{filter}\NormalTok{(}\FunctionTok{year}\NormalTok{(Date)}\SpecialCharTok{\textgreater{}}\DecValTok{1900}\NormalTok{) }\SpecialCharTok{\%\textgreater{}\%}
  \FunctionTok{autoplot}\NormalTok{(LandOceanAvgTemp, }\AttributeTok{color =} \StringTok{\textquotesingle{}gray\textquotesingle{}}\NormalTok{) }\SpecialCharTok{+} 
  \FunctionTok{autolayer}\NormalTok{(comp\_after1900, trend, }\AttributeTok{color =} \StringTok{\textquotesingle{}\#6666ff\textquotesingle{}}\NormalTok{) }\SpecialCharTok{+} 
  \FunctionTok{ggtitle}\NormalTok{(}\StringTok{\textquotesingle{}Land and Ocean Global Temperature Trend\textquotesingle{}}\NormalTok{)}\SpecialCharTok{+}
  \FunctionTok{labs}\NormalTok{(}\AttributeTok{caption =} \StringTok{"fig. 21"}\NormalTok{)}
\end{Highlighting}
\end{Shaded}

\includegraphics{Project_files/figure-latex/unnamed-chunk-12-3.pdf}

\begin{Shaded}
\begin{Highlighting}[]
\NormalTok{land\_global\_temp }\SpecialCharTok{\%\textgreater{}\%}
  \FunctionTok{autoplot}\NormalTok{(LandAvgTemp, }\AttributeTok{color =} \StringTok{\textquotesingle{}gray\textquotesingle{}}\NormalTok{) }\SpecialCharTok{+} 
  \FunctionTok{autolayer}\NormalTok{(}\FunctionTok{components}\NormalTok{(dec2), season\_adjust, }\AttributeTok{color =} \StringTok{\textquotesingle{}purple\textquotesingle{}}\NormalTok{) }\SpecialCharTok{+} 
  \FunctionTok{ggtitle}\NormalTok{(}\StringTok{\textquotesingle{}Seasonal Adjusted Land Global Temperature\textquotesingle{}}\NormalTok{)}\SpecialCharTok{+}
  \FunctionTok{labs}\NormalTok{(}\AttributeTok{caption =} \StringTok{"fig. 22"}\NormalTok{)}
\end{Highlighting}
\end{Shaded}

\includegraphics{Project_files/figure-latex/unnamed-chunk-12-4.pdf}

In the first graph, it is possible to depict a notable growth in land
and ocean temperatures trend over the years. When comparing this graph
with the second one (depicting the Land Global temperatures trend), we
can see that the second one's trend is not growing as significantly.
This may mean that the rising trend in the oceans temperature is
contributing highly to the upward trend in land plus oceans temperature.

In the third graph, zooming in to more recent years, we can see the
evident growth in land and ocean global average temperatures.

In the last graph, after adjusting the series regarding seasonal
variations, we can clearly see that the land average global temperature
is rising, more significantly from 1950 to the current years.

\begin{Shaded}
\begin{Highlighting}[]
\FunctionTok{components}\NormalTok{(dec) }\SpecialCharTok{\%\textgreater{}\%}
  \FunctionTok{autoplot}\NormalTok{() }\SpecialCharTok{+} \FunctionTok{ggtitle}\NormalTok{(}\StringTok{\textquotesingle{}Land and Ocean Global Temperature STL Decomposition\textquotesingle{}}\NormalTok{)}\SpecialCharTok{+}
  \FunctionTok{labs}\NormalTok{(}\AttributeTok{caption =} \StringTok{"fig. 23"}\NormalTok{)}
\end{Highlighting}
\end{Shaded}

\includegraphics{Project_files/figure-latex/unnamed-chunk-13-1.pdf}

\begin{Shaded}
\begin{Highlighting}[]
\CommentTok{\#components(dec2) \%\textgreater{}\%}
\CommentTok{\#  autoplot() + ggtitle(\textquotesingle{}Land Global Temperature\textquotesingle{}) }

\FunctionTok{components}\NormalTok{(dec) }\SpecialCharTok{\%\textgreater{}\%}
  \FunctionTok{autoplot}\NormalTok{(remainder, }\AttributeTok{color =} \StringTok{\textquotesingle{}\#ff668c\textquotesingle{}}\NormalTok{) }\SpecialCharTok{+} \FunctionTok{ggtitle}\NormalTok{(}\StringTok{\textquotesingle{}Land Global Temperature Remainder\textquotesingle{}}\NormalTok{)}\SpecialCharTok{+}
  \FunctionTok{labs}\NormalTok{(}\AttributeTok{caption =} \StringTok{"fig. 24"}\NormalTok{) }
\end{Highlighting}
\end{Shaded}

\includegraphics{Project_files/figure-latex/unnamed-chunk-13-2.pdf}

\begin{Shaded}
\begin{Highlighting}[]
\CommentTok{\#components(dec2) \%\textgreater{}\%}
\CommentTok{\#autoplot(remainder, color = \textquotesingle{}\#8c66ff\textquotesingle{}) + ggtitle(\textquotesingle{}Land and Ocean Global Temperature Remainder\textquotesingle{}) }
\end{Highlighting}
\end{Shaded}

How is STL for our time series?

Looking at the remainder, there is no clear pattern. That is a sign that
the STL model is capturing our season and our trend in an appropriate
manner.

\#\#Decomposing using SEATS Decomposition

\begin{Shaded}
\begin{Highlighting}[]
\NormalTok{land\_ocean\_global\_temp }\SpecialCharTok{\%\textgreater{}\%} 
  \FunctionTok{filter}\NormalTok{(}\FunctionTok{year}\NormalTok{(Date) }\SpecialCharTok{\textgreater{}} \DecValTok{1990}\NormalTok{) }\SpecialCharTok{\%\textgreater{}\%}
  \FunctionTok{model}\NormalTok{(}\AttributeTok{seats =}\NormalTok{ feasts}\SpecialCharTok{:::} \FunctionTok{SEATS}\NormalTok{(LandOceanAvgTemp)) }\SpecialCharTok{\%\textgreater{}\%}
  \FunctionTok{components}\NormalTok{() }\OtherTok{{-}\textgreater{}}\NormalTok{ seats\_dec}

\FunctionTok{autoplot}\NormalTok{(seats\_dec) }\SpecialCharTok{+} \FunctionTok{labs}\NormalTok{(}\AttributeTok{title =} \StringTok{"Land and Ocean Global Temperature X{-}13 ARIMA{-}SEATS Decomposition"}\NormalTok{, }\AttributeTok{caption =} \StringTok{"fig. 25"}\NormalTok{)}
\end{Highlighting}
\end{Shaded}

\includegraphics{Project_files/figure-latex/unnamed-chunk-14-1.pdf}

Did SEATS help?

The trend is smoother. It is easier to see that the trend is going up.

\hypertarget{trend-in-different-continents}{%
\subsection{Trend in different
Continents}\label{trend-in-different-continents}}

\begin{Shaded}
\begin{Highlighting}[]
\CommentTok{\#{-}{-}{-}{-}{-}{-}{-}{-}{-}{-}{-}{-}{-}{-}{-}{-}{-}{-}{-}{-}{-}{-}{-}{-}{-}{-}{-}{-}{-}{-}{-}{-}{-}{-}{-}{-}{-}{-}{-}{-}{-}{-}{-}{-}{-}{-}{-}{-}{-}{-}{-}{-}{-}{-}{-}{-}{-}{-}{-}{-}{-}{-}{-}{-}}

\NormalTok{GlobalCountryTemp }\SpecialCharTok{\%\textgreater{}\%}
  \FunctionTok{filter}\NormalTok{(}\FunctionTok{year}\NormalTok{(Date)}\SpecialCharTok{\textgreater{}} \DecValTok{1850}\NormalTok{) }\SpecialCharTok{\%\textgreater{}\%}
  \FunctionTok{filter}\NormalTok{ (Country }\SpecialCharTok{==} \StringTok{\textquotesingle{}Europe\textquotesingle{}}\NormalTok{) }\SpecialCharTok{\%\textgreater{}\%}
  \FunctionTok{select}\NormalTok{(LandAvgTemp) }\SpecialCharTok{\%\textgreater{}\%}
  \FunctionTok{as\_tibble}\NormalTok{() }\SpecialCharTok{\%\textgreater{}\%}
  \FunctionTok{group\_by}\NormalTok{(Date) }\SpecialCharTok{\%\textgreater{}\%} 
  \FunctionTok{summarise\_at}\NormalTok{(}\FunctionTok{vars}\NormalTok{(LandAvgTemp), }\FunctionTok{list}\NormalTok{(}\AttributeTok{LandTempEurope =}\NormalTok{ mean)) }\SpecialCharTok{\%\textgreater{}\%}
  \FunctionTok{as\_tsibble}\NormalTok{() }\SpecialCharTok{\%\textgreater{}\%}
  \FunctionTok{fill}\NormalTok{(LandTempEurope, }\AttributeTok{.direction =} \StringTok{\textquotesingle{}down\textquotesingle{}}\NormalTok{) }\OtherTok{{-}\textgreater{}}\NormalTok{ tempEurope}

\NormalTok{decEurope }\OtherTok{\textless{}{-}}\NormalTok{ tempEurope }\SpecialCharTok{\%\textgreater{}\%} 
  \FunctionTok{model}\NormalTok{(}\AttributeTok{stl =} \FunctionTok{STL}\NormalTok{(LandTempEurope))}

\CommentTok{\#components(decEurope)}

\NormalTok{tempEurope }\SpecialCharTok{\%\textgreater{}\%}
  \FunctionTok{autoplot}\NormalTok{(LandTempEurope, }\AttributeTok{color =} \StringTok{\textquotesingle{}gray\textquotesingle{}}\NormalTok{) }\SpecialCharTok{+} 
  \FunctionTok{autolayer}\NormalTok{(}\FunctionTok{components}\NormalTok{(decEurope), trend, }\AttributeTok{color =} \StringTok{\textquotesingle{}\#9900cc\textquotesingle{}}\NormalTok{) }\SpecialCharTok{+} 
  \FunctionTok{ggtitle}\NormalTok{(}\StringTok{\textquotesingle{}Land Temperature Europe Trend\textquotesingle{}}\NormalTok{) }\SpecialCharTok{+}
  \FunctionTok{labs}\NormalTok{(}\AttributeTok{caption =} \StringTok{"fig. 26"}\NormalTok{)}
\end{Highlighting}
\end{Shaded}

\includegraphics{Project_files/figure-latex/unnamed-chunk-15-1.pdf}

\begin{Shaded}
\begin{Highlighting}[]
\CommentTok{\#{-}{-}{-}{-}{-}{-}{-}{-}{-}{-}{-}{-}{-}{-}{-}{-}{-}{-}{-}{-}{-}{-}{-}{-}{-}{-}{-}{-}{-}{-}{-}{-}{-}{-}{-}{-}{-}{-}{-}{-}{-}{-}{-}{-}{-}{-}{-}{-}{-}{-}{-}{-}{-}{-}{-}{-}{-}{-}{-}{-}{-}{-}{-}{-}}

\NormalTok{GlobalCountryTemp }\SpecialCharTok{\%\textgreater{}\%}
  \FunctionTok{filter}\NormalTok{(}\FunctionTok{year}\NormalTok{(Date)}\SpecialCharTok{\textgreater{}} \DecValTok{1860}\NormalTok{) }\SpecialCharTok{\%\textgreater{}\%}
  \FunctionTok{filter}\NormalTok{ (Country }\SpecialCharTok{==} \StringTok{\textquotesingle{}Africa\textquotesingle{}}\NormalTok{) }\SpecialCharTok{\%\textgreater{}\%}
  \FunctionTok{select}\NormalTok{(LandAvgTemp) }\SpecialCharTok{\%\textgreater{}\%}
  \FunctionTok{as\_tibble}\NormalTok{() }\SpecialCharTok{\%\textgreater{}\%}
  \FunctionTok{group\_by}\NormalTok{(Date) }\SpecialCharTok{\%\textgreater{}\%} 
  \FunctionTok{summarise\_at}\NormalTok{(}\FunctionTok{vars}\NormalTok{(LandAvgTemp), }\FunctionTok{list}\NormalTok{(}\AttributeTok{LandTempAfrica =}\NormalTok{ mean)) }\SpecialCharTok{\%\textgreater{}\%}
  \FunctionTok{as\_tsibble}\NormalTok{() }\SpecialCharTok{\%\textgreater{}\%}
  \FunctionTok{fill}\NormalTok{(LandTempAfrica, }\AttributeTok{.direction =} \StringTok{\textquotesingle{}down\textquotesingle{}}\NormalTok{) }\OtherTok{{-}\textgreater{}}\NormalTok{ tempAfrica}

\NormalTok{decAfrica }\OtherTok{\textless{}{-}}\NormalTok{ tempAfrica }\SpecialCharTok{\%\textgreater{}\%} 
  \FunctionTok{model}\NormalTok{(}\AttributeTok{stl =} \FunctionTok{STL}\NormalTok{(LandTempAfrica))}

\CommentTok{\#components(decAfrica)}

\NormalTok{tempAfrica }\SpecialCharTok{\%\textgreater{}\%}
  \FunctionTok{autoplot}\NormalTok{(LandTempAfrica, }\AttributeTok{color =} \StringTok{\textquotesingle{}gray\textquotesingle{}}\NormalTok{) }\SpecialCharTok{+} 
  \FunctionTok{autolayer}\NormalTok{(}\FunctionTok{components}\NormalTok{(decAfrica), trend, }\AttributeTok{color =} \StringTok{\textquotesingle{}\#e65c00\textquotesingle{}}\NormalTok{) }\SpecialCharTok{+} 
  \FunctionTok{ggtitle}\NormalTok{(}\StringTok{\textquotesingle{}Land Temperature Africa Trend\textquotesingle{}}\NormalTok{) }\SpecialCharTok{+}
  \FunctionTok{labs}\NormalTok{(}\AttributeTok{caption =} \StringTok{"fig. 27"}\NormalTok{)}
\end{Highlighting}
\end{Shaded}

\includegraphics{Project_files/figure-latex/unnamed-chunk-15-2.pdf}

\begin{Shaded}
\begin{Highlighting}[]
\CommentTok{\#{-}{-}{-}{-}{-}{-}{-}{-}{-}{-}{-}{-}{-}{-}{-}{-}{-}{-}{-}{-}{-}{-}{-}{-}{-}{-}{-}{-}{-}{-}{-}{-}{-}{-}{-}{-}{-}{-}{-}{-}{-}{-}{-}{-}{-}{-}{-}{-}{-}{-}{-}{-}{-}{-}{-}{-}{-}{-}{-}{-}{-}{-}{-}{-}}

\NormalTok{GlobalCountryTemp }\SpecialCharTok{\%\textgreater{}\%}
  \FunctionTok{filter}\NormalTok{(}\FunctionTok{year}\NormalTok{(Date)}\SpecialCharTok{\textgreater{}} \DecValTok{1850}\NormalTok{) }\SpecialCharTok{\%\textgreater{}\%}
  \FunctionTok{filter}\NormalTok{ (Country }\SpecialCharTok{==} \StringTok{\textquotesingle{}Asia\textquotesingle{}}\NormalTok{) }\SpecialCharTok{\%\textgreater{}\%}
  \FunctionTok{select}\NormalTok{(LandAvgTemp) }\SpecialCharTok{\%\textgreater{}\%}
  \FunctionTok{as\_tibble}\NormalTok{() }\SpecialCharTok{\%\textgreater{}\%}
  \FunctionTok{group\_by}\NormalTok{(Date) }\SpecialCharTok{\%\textgreater{}\%} 
  \FunctionTok{summarise\_at}\NormalTok{(}\FunctionTok{vars}\NormalTok{(LandAvgTemp), }\FunctionTok{list}\NormalTok{(}\AttributeTok{LandTempAsia =}\NormalTok{ mean)) }\SpecialCharTok{\%\textgreater{}\%}
  \FunctionTok{as\_tsibble}\NormalTok{() }\SpecialCharTok{\%\textgreater{}\%}
  \FunctionTok{fill}\NormalTok{(LandTempAsia, }\AttributeTok{.direction =} \StringTok{\textquotesingle{}down\textquotesingle{}}\NormalTok{) }\OtherTok{{-}\textgreater{}}\NormalTok{ tempAsia}

\NormalTok{decAsia }\OtherTok{\textless{}{-}}\NormalTok{ tempAsia }\SpecialCharTok{\%\textgreater{}\%} 
  \FunctionTok{model}\NormalTok{(}\AttributeTok{stl =} \FunctionTok{STL}\NormalTok{(LandTempAsia))}

\CommentTok{\#components(decAsia)}

\NormalTok{tempAsia }\SpecialCharTok{\%\textgreater{}\%}
  \FunctionTok{autoplot}\NormalTok{(LandTempAsia, }\AttributeTok{color =} \StringTok{\textquotesingle{}gray\textquotesingle{}}\NormalTok{) }\SpecialCharTok{+} 
  \FunctionTok{autolayer}\NormalTok{(}\FunctionTok{components}\NormalTok{(decAsia), trend, }\AttributeTok{color =} \StringTok{\textquotesingle{}\#ff4d94\textquotesingle{}}\NormalTok{) }\SpecialCharTok{+} 
  \FunctionTok{ggtitle}\NormalTok{(}\StringTok{\textquotesingle{}Land Temperature Asia Trend\textquotesingle{}}\NormalTok{) }\SpecialCharTok{+}
  \FunctionTok{labs}\NormalTok{(}\AttributeTok{caption =} \StringTok{"fig. 28"}\NormalTok{)}
\end{Highlighting}
\end{Shaded}

\includegraphics{Project_files/figure-latex/unnamed-chunk-15-3.pdf}

\begin{Shaded}
\begin{Highlighting}[]
\CommentTok{\#{-}{-}{-}{-}{-}{-}{-}{-}{-}{-}{-}{-}{-}{-}{-}{-}{-}{-}{-}{-}{-}{-}{-}{-}{-}{-}{-}{-}{-}{-}{-}{-}{-}{-}{-}{-}{-}{-}{-}{-}{-}{-}{-}{-}{-}{-}{-}{-}{-}{-}{-}{-}{-}{-}{-}{-}{-}{-}{-}{-}{-}{-}{-}{-}}

\NormalTok{GlobalCountryTemp }\SpecialCharTok{\%\textgreater{}\%}
  \FunctionTok{filter}\NormalTok{(}\FunctionTok{year}\NormalTok{(Date)}\SpecialCharTok{\textgreater{}} \DecValTok{1850}\NormalTok{) }\SpecialCharTok{\%\textgreater{}\%}
  \FunctionTok{filter}\NormalTok{ (Country }\SpecialCharTok{==} \StringTok{\textquotesingle{}South America\textquotesingle{}}\NormalTok{) }\SpecialCharTok{\%\textgreater{}\%}
  \FunctionTok{select}\NormalTok{(LandAvgTemp) }\SpecialCharTok{\%\textgreater{}\%}
  \FunctionTok{as\_tibble}\NormalTok{() }\SpecialCharTok{\%\textgreater{}\%}
  \FunctionTok{group\_by}\NormalTok{(Date) }\SpecialCharTok{\%\textgreater{}\%} 
  \FunctionTok{summarise\_at}\NormalTok{(}\FunctionTok{vars}\NormalTok{(LandAvgTemp), }\FunctionTok{list}\NormalTok{(}\AttributeTok{LandTempSouthAmerica =}\NormalTok{ mean)) }\SpecialCharTok{\%\textgreater{}\%}
  \FunctionTok{as\_tsibble}\NormalTok{() }\SpecialCharTok{\%\textgreater{}\%}
  \FunctionTok{fill}\NormalTok{(LandTempSouthAmerica, }\AttributeTok{.direction =} \StringTok{\textquotesingle{}down\textquotesingle{}}\NormalTok{) }\OtherTok{{-}\textgreater{}}\NormalTok{ tempSouthAmerica}

\NormalTok{decSouthAmerica }\OtherTok{\textless{}{-}}\NormalTok{ tempSouthAmerica }\SpecialCharTok{\%\textgreater{}\%} 
  \FunctionTok{model}\NormalTok{(}\AttributeTok{stl =} \FunctionTok{STL}\NormalTok{(LandTempSouthAmerica))}

\CommentTok{\#components(decSouthAmerica)}

\NormalTok{tempSouthAmerica }\SpecialCharTok{\%\textgreater{}\%}
  \FunctionTok{autoplot}\NormalTok{(LandTempSouthAmerica, }\AttributeTok{color =} \StringTok{\textquotesingle{}gray\textquotesingle{}}\NormalTok{) }\SpecialCharTok{+} 
  \FunctionTok{autolayer}\NormalTok{(}\FunctionTok{components}\NormalTok{(decSouthAmerica), trend, }\AttributeTok{color =} \StringTok{\textquotesingle{}\#008040\textquotesingle{}}\NormalTok{) }\SpecialCharTok{+} 
  \FunctionTok{ggtitle}\NormalTok{(}\StringTok{\textquotesingle{}Land Temperature South America Trend\textquotesingle{}}\NormalTok{) }\SpecialCharTok{+}
  \FunctionTok{labs}\NormalTok{(}\AttributeTok{caption =} \StringTok{"fig. 29"}\NormalTok{)}
\end{Highlighting}
\end{Shaded}

\includegraphics{Project_files/figure-latex/unnamed-chunk-15-4.pdf}

\begin{Shaded}
\begin{Highlighting}[]
\CommentTok{\#{-}{-}{-}{-}{-}{-}{-}{-}{-}{-}{-}{-}{-}{-}{-}{-}{-}{-}{-}{-}{-}{-}{-}{-}{-}{-}{-}{-}{-}{-}{-}{-}{-}{-}{-}{-}{-}{-}{-}{-}{-}{-}{-}{-}{-}{-}{-}{-}{-}{-}{-}{-}{-}{-}{-}{-}{-}{-}{-}{-}{-}{-}{-}{-}}

\NormalTok{GlobalCountryTemp }\SpecialCharTok{\%\textgreater{}\%}
  \FunctionTok{filter}\NormalTok{(}\FunctionTok{year}\NormalTok{(Date)}\SpecialCharTok{\textgreater{}} \DecValTok{1850}\NormalTok{) }\SpecialCharTok{\%\textgreater{}\%}
  \FunctionTok{filter}\NormalTok{ (Country }\SpecialCharTok{==} \StringTok{\textquotesingle{}North America\textquotesingle{}}\NormalTok{) }\SpecialCharTok{\%\textgreater{}\%}
  \FunctionTok{select}\NormalTok{(LandAvgTemp) }\SpecialCharTok{\%\textgreater{}\%}
  \FunctionTok{as\_tibble}\NormalTok{() }\SpecialCharTok{\%\textgreater{}\%}
  \FunctionTok{group\_by}\NormalTok{(Date) }\SpecialCharTok{\%\textgreater{}\%} 
  \FunctionTok{summarise\_at}\NormalTok{(}\FunctionTok{vars}\NormalTok{(LandAvgTemp), }\FunctionTok{list}\NormalTok{(}\AttributeTok{LandTempNorthAmerica =}\NormalTok{ mean)) }\SpecialCharTok{\%\textgreater{}\%}
  \FunctionTok{as\_tsibble}\NormalTok{() }\SpecialCharTok{\%\textgreater{}\%}
  \FunctionTok{fill}\NormalTok{(LandTempNorthAmerica, }\AttributeTok{.direction =} \StringTok{\textquotesingle{}down\textquotesingle{}}\NormalTok{) }\OtherTok{{-}\textgreater{}}\NormalTok{ tempNorthAmerica}

\NormalTok{decNorthAmerica }\OtherTok{\textless{}{-}}\NormalTok{ tempNorthAmerica }\SpecialCharTok{\%\textgreater{}\%} 
  \FunctionTok{model}\NormalTok{(}\AttributeTok{stl =} \FunctionTok{STL}\NormalTok{(LandTempNorthAmerica))}

\CommentTok{\#components(decNorthAmerica)}

\NormalTok{tempNorthAmerica }\SpecialCharTok{\%\textgreater{}\%}
  \FunctionTok{autoplot}\NormalTok{(LandTempNorthAmerica, }\AttributeTok{color =} \StringTok{\textquotesingle{}gray\textquotesingle{}}\NormalTok{) }\SpecialCharTok{+} 
  \FunctionTok{autolayer}\NormalTok{(}\FunctionTok{components}\NormalTok{(decNorthAmerica), trend, }\AttributeTok{color =} \StringTok{\textquotesingle{}\#009999\textquotesingle{}}\NormalTok{) }\SpecialCharTok{+} 
  \FunctionTok{ggtitle}\NormalTok{(}\StringTok{\textquotesingle{}Land Temperature North America Trend\textquotesingle{}}\NormalTok{) }\SpecialCharTok{+}
  \FunctionTok{labs}\NormalTok{(}\AttributeTok{caption =} \StringTok{"fig. 30"}\NormalTok{)}
\end{Highlighting}
\end{Shaded}

\includegraphics{Project_files/figure-latex/unnamed-chunk-15-5.pdf}

\begin{Shaded}
\begin{Highlighting}[]
\CommentTok{\#{-}{-}{-}{-}{-}{-}{-}{-}{-}{-}{-}{-}{-}{-}{-}{-}{-}{-}{-}{-}{-}{-}{-}{-}{-}{-}{-}{-}{-}{-}{-}{-}{-}{-}{-}{-}{-}{-}{-}{-}{-}{-}{-}{-}{-}{-}{-}{-}{-}{-}{-}{-}{-}{-}{-}{-}{-}{-}{-}{-}{-}{-}{-}{-}}

\NormalTok{GlobalCountryTemp }\SpecialCharTok{\%\textgreater{}\%}
  \FunctionTok{filter}\NormalTok{(}\FunctionTok{year}\NormalTok{(Date)}\SpecialCharTok{\textgreater{}} \DecValTok{1850}\NormalTok{) }\SpecialCharTok{\%\textgreater{}\%}
  \FunctionTok{filter}\NormalTok{ (Country }\SpecialCharTok{==} \StringTok{\textquotesingle{}Oceania\textquotesingle{}}\NormalTok{) }\SpecialCharTok{\%\textgreater{}\%}
  \FunctionTok{select}\NormalTok{(LandAvgTemp) }\SpecialCharTok{\%\textgreater{}\%}
  \FunctionTok{as\_tibble}\NormalTok{() }\SpecialCharTok{\%\textgreater{}\%}
  \FunctionTok{group\_by}\NormalTok{(Date) }\SpecialCharTok{\%\textgreater{}\%} 
  \FunctionTok{summarise\_at}\NormalTok{(}\FunctionTok{vars}\NormalTok{(LandAvgTemp), }\FunctionTok{list}\NormalTok{(}\AttributeTok{LandTempOceania =}\NormalTok{ mean)) }\SpecialCharTok{\%\textgreater{}\%}
  \FunctionTok{as\_tsibble}\NormalTok{() }\SpecialCharTok{\%\textgreater{}\%}
  \FunctionTok{fill}\NormalTok{(LandTempOceania, }\AttributeTok{.direction =} \StringTok{\textquotesingle{}down\textquotesingle{}}\NormalTok{) }\OtherTok{{-}\textgreater{}}\NormalTok{ tempOceania}

\NormalTok{decOceania }\OtherTok{\textless{}{-}}\NormalTok{ tempOceania }\SpecialCharTok{\%\textgreater{}\%} 
  \FunctionTok{model}\NormalTok{(}\AttributeTok{stl =} \FunctionTok{STL}\NormalTok{(LandTempOceania))}

\CommentTok{\#components(decOceania)}

\NormalTok{tempOceania }\SpecialCharTok{\%\textgreater{}\%}
  \FunctionTok{autoplot}\NormalTok{(LandTempOceania, }\AttributeTok{color =} \StringTok{\textquotesingle{}gray\textquotesingle{}}\NormalTok{) }\SpecialCharTok{+} 
  \FunctionTok{autolayer}\NormalTok{(}\FunctionTok{components}\NormalTok{(decOceania), trend, }\AttributeTok{color =} \StringTok{\textquotesingle{}\#99004d\textquotesingle{}}\NormalTok{) }\SpecialCharTok{+} 
  \FunctionTok{ggtitle}\NormalTok{(}\StringTok{\textquotesingle{}Land Temperature Oceania Trend\textquotesingle{}}\NormalTok{) }\SpecialCharTok{+}
  \FunctionTok{labs}\NormalTok{(}\AttributeTok{caption =} \StringTok{"fig. 31"}\NormalTok{)}
\end{Highlighting}
\end{Shaded}

\includegraphics{Project_files/figure-latex/unnamed-chunk-15-6.pdf}
Analyzing trends on different continents, we can reach some conclusions:

It is important to make the distinction, that the trend over time is
ranging from very different values from continent to continent. While in
North America the temperatures are much colder in general, in Europe and
Asia they are slight more moderate. In Oceania the temperatures are
hotter, ranging always above 20Cº, but in Africa and South America the
temperatures are alarmingly higher, especially in Africa where
temperatures start at about 23ºC in 1860, and have already over-passed
the 25ºC.

In Africa and South America the trend is going upwards significantly,
and much faster, compared to other continents. In Asia and North
America, the trend is increasing, but more slowly than in other
continents.

\hypertarget{trend-and-seasonal-strenght-in-different-countries-and-continents}{%
\subsection{Trend and Seasonal Strenght in different Countries and
Continents}\label{trend-and-seasonal-strenght-in-different-countries-and-continents}}

\begin{Shaded}
\begin{Highlighting}[]
\NormalTok{GlobalCountryTemp }\SpecialCharTok{\%\textgreater{}\%} 
  \FunctionTok{fill}\NormalTok{(LandAvgTemp, }\AttributeTok{.direction =} \StringTok{"down"}\NormalTok{) }\SpecialCharTok{\%\textgreater{}\%}
  \FunctionTok{features}\NormalTok{(LandAvgTemp, feat\_stl) }\SpecialCharTok{\%\textgreater{}\%}
  \FunctionTok{drop\_na}\NormalTok{(trend\_strength, seasonal\_strength\_year) }\SpecialCharTok{\%\textgreater{}\%}
  \FunctionTok{mutate}\NormalTok{(}\AttributeTok{Continent =} \FunctionTok{case\_when}\NormalTok{(}
\NormalTok{    Country }\SpecialCharTok{\%in\%}\NormalTok{ europe  }\SpecialCharTok{\textasciitilde{}} \StringTok{"Europe"}\NormalTok{ ,}
\NormalTok{    Country }\SpecialCharTok{\%in\%}\NormalTok{ africa  }\SpecialCharTok{\textasciitilde{}} \StringTok{"Africa"}\NormalTok{ ,}
\NormalTok{    Country }\SpecialCharTok{\%in\%}\NormalTok{ asia  }\SpecialCharTok{\textasciitilde{}} \StringTok{"Asia"}\NormalTok{ ,}
\NormalTok{    Country }\SpecialCharTok{\%in\%}\NormalTok{ north\_america  }\SpecialCharTok{\textasciitilde{}} \StringTok{"North America"}\NormalTok{,}
\NormalTok{    Country }\SpecialCharTok{\%in\%}\NormalTok{ south\_america  }\SpecialCharTok{\textasciitilde{}} \StringTok{"South America"}\NormalTok{,}
\NormalTok{    Country }\SpecialCharTok{\%in\%}\NormalTok{ oceania  }\SpecialCharTok{\textasciitilde{}} \StringTok{"Oceania"}
\NormalTok{  )}
\NormalTok{) }\SpecialCharTok{\%\textgreater{}\%}
  \FunctionTok{drop\_na}\NormalTok{(Continent) }\SpecialCharTok{\%\textgreater{}\%}
  \FunctionTok{ggplot}\NormalTok{(}\FunctionTok{aes}\NormalTok{(}\AttributeTok{x =}\NormalTok{ trend\_strength, }\AttributeTok{y =}\NormalTok{ seasonal\_strength\_year, }\AttributeTok{colour =}\NormalTok{ Continent)) }\SpecialCharTok{+}
  \FunctionTok{scale\_color\_manual}\NormalTok{(}\AttributeTok{values=}\FunctionTok{c}\NormalTok{(}\StringTok{\textquotesingle{}\#ff9900\textquotesingle{}}\NormalTok{, }\StringTok{\textquotesingle{}\#ff4d94\textquotesingle{}}\NormalTok{, }\StringTok{\textquotesingle{}\#9900cc\textquotesingle{}}\NormalTok{, }\StringTok{\textquotesingle{}\#009999\textquotesingle{}}\NormalTok{, }\StringTok{\textquotesingle{}\#ffcc00\textquotesingle{}}\NormalTok{, }\StringTok{\textquotesingle{}\#00cc00\textquotesingle{}}\NormalTok{)) }\SpecialCharTok{+}
  \FunctionTok{geom\_point}\NormalTok{() }\SpecialCharTok{+}
  \FunctionTok{labs}\NormalTok{(}\AttributeTok{title=} \StringTok{\textquotesingle{}Trend and Seasonal Strenght in different Countries\textquotesingle{}}\NormalTok{, }\AttributeTok{caption =} \StringTok{"fig. 32"}\NormalTok{)}
\end{Highlighting}
\end{Shaded}

\includegraphics{Project_files/figure-latex/unnamed-chunk-16-1.pdf}

In the scatter plot above, we can evaluate the strength of the trends
and seasonal patterns on different countries, grouped by continent.

In all countries in the graph, the seasonal patterns are clearly strong
(the majority with a seasonal strength higher than 0.7).

In Europe, we can see countries are in a cluster with a very high
seasonal strength (the differences in temperature according to months,
are more pronounced) and lower trend strength (it is not going on an
upward trend as much).

In Asia the seasonal strength is very high in almost all plotted
countries (above 0.85). Meaning the differences in temperatures in
distinct seasons are very significant (colder winters and very hot
summers/spring).

In North America, the trend is very string in most of the countries,
meaning temperatures are rising quite fast and significantly.

\hypertarget{subseries-of-land-and-ocean-global-temperature}{%
\subsection{Subseries of Land and Ocean Global
Temperature}\label{subseries-of-land-and-ocean-global-temperature}}

\begin{Shaded}
\begin{Highlighting}[]
\FunctionTok{components}\NormalTok{(dec) }\SpecialCharTok{\%\textgreater{}\%} \FunctionTok{gg\_subseries}\NormalTok{(season\_year) }\SpecialCharTok{+} 
  \FunctionTok{labs}\NormalTok{(}\AttributeTok{title =} \StringTok{\textquotesingle{}Subseries by month of Land and Ocean Global Temperatures\textquotesingle{}}\NormalTok{, }\AttributeTok{caption =} \StringTok{"fig. 33"}\NormalTok{) }
\end{Highlighting}
\end{Shaded}

\includegraphics{Project_files/figure-latex/unnamed-chunk-17-1.pdf}

Moreover, on the next plot we can see that on average both the global
land temperature and the global land and ocean temperature are much
higher in Summer months, compared to the rest, and much lower in
January, February, and December. Interestingly in July and August the
trend is going down, so in previous years, temperatures were higher in
these months, compared to recent years.

\hypertarget{the-forecasters-toolbox}{%
\subsection{4. The forecaster's toolbox}\label{the-forecasters-toolbox}}

In the previous chapters we have already prepared and visualized our
data in order to gain a good understanding of our data. Now, we will use
some benchmark forecasting methods to forecast our time series.
Moreover, methods for assessing forecast accuracy will be applied.

\hypertarget{specifying-a-model}{%
\subsubsection{Specifying a model}\label{specifying-a-model}}

Looking at our data allows us to identify common patterns, and
subsequently specify an appropriate model. We have seen in the previous
chapters that our time series for Land Average Temperature and Land \&
Ocean Average temperature reveal a seasonal and a trend pattern. Now, we
will check which type of forecasting method is appropriate for this type
of time series.

There are several simple forecasting methods. The ``Average method'' and
the ``Naïve method'' are suitable for time series without any
seasonality or trend pattern. Therefore, we won't use these methods for
our time series.

The ``Seasonal naïve method'' is appropriate for data with seasonality
and the ``Drift Method'' for data with a trend pattern. For that reason,
we will know apply these forecasting methods to our data and visualize
the results.

\begin{Shaded}
\begin{Highlighting}[]
\NormalTok{GlobalCountryTemp }\SpecialCharTok{\%\textgreater{}\%} 
  \FunctionTok{filter}\NormalTok{(Country}\SpecialCharTok{==}\StringTok{"Global"}\NormalTok{) }\SpecialCharTok{\%\textgreater{}\%} 
  \FunctionTok{fill}\NormalTok{(LandAvgTemp, }\AttributeTok{.direction =} \StringTok{"down"}\NormalTok{) }\SpecialCharTok{\%\textgreater{}\%} 
  \FunctionTok{model}\NormalTok{(}\AttributeTok{Seasonal\_naive =} \FunctionTok{SNAIVE}\NormalTok{(LandAvgTemp),}
        \AttributeTok{Drift =} \FunctionTok{RW}\NormalTok{(LandAvgTemp }\SpecialCharTok{\textasciitilde{}} \FunctionTok{drift}\NormalTok{())}
\NormalTok{        ) }\SpecialCharTok{\%\textgreater{}\%} \FunctionTok{forecast}\NormalTok{(}\AttributeTok{h =} \StringTok{"8 years"}\NormalTok{) }\SpecialCharTok{\%\textgreater{}\%} 
  \FunctionTok{autoplot}\NormalTok{((}\FunctionTok{filter}\NormalTok{(GlobalCountryTemp, }\FunctionTok{year}\NormalTok{(Date) }\SpecialCharTok{\textgreater{}} \DecValTok{2000}\NormalTok{))) }\SpecialCharTok{+} 
  \FunctionTok{labs}\NormalTok{(}\AttributeTok{title =} \StringTok{"Global Land Average Temperature {-} Forecast"}\NormalTok{,}
       \AttributeTok{subtitle =} \StringTok{"Drift and Seasonal naïve method"}\NormalTok{,}
       \AttributeTok{y=}\StringTok{"LandAvgTemp Cº"}\NormalTok{,}
       \AttributeTok{caption =} \StringTok{"fig. 34"}\NormalTok{)}
\end{Highlighting}
\end{Shaded}

\includegraphics{Project_files/figure-latex/unnamed-chunk-18-1.pdf}

\begin{Shaded}
\begin{Highlighting}[]
\NormalTok{GlobalCountryTemp }\SpecialCharTok{\%\textgreater{}\%} 
  \FunctionTok{filter}\NormalTok{(Country}\SpecialCharTok{==}\StringTok{"Global"}\NormalTok{) }\SpecialCharTok{\%\textgreater{}\%} 
  \FunctionTok{fill}\NormalTok{(LandOceanAvgTemp, }\AttributeTok{.direction =} \StringTok{"down"}\NormalTok{) }\SpecialCharTok{\%\textgreater{}\%} 
  \FunctionTok{model}\NormalTok{(}\AttributeTok{Seasonal\_naive =} \FunctionTok{SNAIVE}\NormalTok{(LandOceanAvgTemp),}
        \AttributeTok{Drift =} \FunctionTok{RW}\NormalTok{(LandOceanAvgTemp }\SpecialCharTok{\textasciitilde{}} \FunctionTok{drift}\NormalTok{())}
\NormalTok{        ) }\SpecialCharTok{\%\textgreater{}\%} \FunctionTok{forecast}\NormalTok{(}\AttributeTok{h =} \StringTok{"8 years"}\NormalTok{) }\SpecialCharTok{\%\textgreater{}\%} 
  \FunctionTok{autoplot}\NormalTok{(}\FunctionTok{filter}\NormalTok{(GlobalCountryTemp, }\FunctionTok{year}\NormalTok{(Date) }\SpecialCharTok{\textgreater{}} \DecValTok{2000}\NormalTok{)) }\SpecialCharTok{+}
  \FunctionTok{labs}\NormalTok{(}\AttributeTok{title =} \StringTok{"Global Land \& Ocean Average Temperature {-} Forecast"}\NormalTok{,}
       \AttributeTok{subtitle =} \StringTok{"Drift and Seasonal naïve method"}\NormalTok{,}
       \AttributeTok{y=}\StringTok{"LandOceanAvgTemp Cº"}\NormalTok{,}
       \AttributeTok{caption =} \StringTok{"fig. 35"}\NormalTok{)}
\end{Highlighting}
\end{Shaded}

\includegraphics{Project_files/figure-latex/unnamed-chunk-19-1.pdf}

The figures above show the forecasts for the next 8 years (from 2016
onward) estimated by the ``Seasonal naïve method'' and the ``Drift
Method'' for the times Series ``Global Land Average temperature'' and
``Global Land \& Ocean Temperature'', respectively. It can be seen that
the area for the prediction intervals is high (espicially for the
``Drift method''), which means that there is a high uncertainty in the
forecasts. One reason for this may be that one method is suitable for
data with seasonality and the other method for data with trend. But in
fact our time series has both seasonality and trend.

Let's try to apply the ``Drift Method'' on the seasonal adjusted data,
so that we capture both components in our forecast: Seasonality and
Trend.

\begin{Shaded}
\begin{Highlighting}[]
\NormalTok{GlobalCountryTemp }\SpecialCharTok{\%\textgreater{}\%} 
  \FunctionTok{filter}\NormalTok{(Country}\SpecialCharTok{==}\StringTok{"Global"}\NormalTok{) }\SpecialCharTok{\%\textgreater{}\%} 
  \FunctionTok{fill}\NormalTok{(LandAvgTemp, }\AttributeTok{.direction =} \StringTok{"down"}\NormalTok{) }\SpecialCharTok{\%\textgreater{}\%}
  \FunctionTok{model}\NormalTok{(}\AttributeTok{stlf =} \FunctionTok{decomposition\_model}\NormalTok{(}
    \FunctionTok{STL}\NormalTok{(LandAvgTemp }\SpecialCharTok{\textasciitilde{}} \FunctionTok{trend}\NormalTok{(}\AttributeTok{window =} \DecValTok{7}\NormalTok{), }\AttributeTok{robust =} \ConstantTok{TRUE}\NormalTok{),}
        \AttributeTok{Drift =} \FunctionTok{RW}\NormalTok{(season\_adjust }\SpecialCharTok{\textasciitilde{}} \FunctionTok{drift}\NormalTok{())}
\NormalTok{    )) }\SpecialCharTok{\%\textgreater{}\%} 
  \FunctionTok{forecast}\NormalTok{(}\AttributeTok{h =} \StringTok{"8 years"}\NormalTok{) }\SpecialCharTok{\%\textgreater{}\%} 
  \FunctionTok{autoplot}\NormalTok{(}\FunctionTok{filter}\NormalTok{(GlobalCountryTemp, }\FunctionTok{year}\NormalTok{(Date) }\SpecialCharTok{\textgreater{}} \DecValTok{2000}\NormalTok{)) }\SpecialCharTok{+}
  \FunctionTok{labs}\NormalTok{(}\AttributeTok{title =} \StringTok{"Global Land Average Temperature {-} Forecast"}\NormalTok{,}
       \AttributeTok{subtitle =} \StringTok{"Drift method based on seasonal adjusted data"}\NormalTok{,}
       \AttributeTok{y=}\StringTok{"LandAvgTemp Cº"}\NormalTok{,}
       \AttributeTok{caption =} \StringTok{"fig. 36"}\NormalTok{)}
\end{Highlighting}
\end{Shaded}

\includegraphics{Project_files/figure-latex/unnamed-chunk-20-1.pdf}

\begin{Shaded}
\begin{Highlighting}[]
\NormalTok{GlobalCountryTemp }\SpecialCharTok{\%\textgreater{}\%} 
  \FunctionTok{filter}\NormalTok{(Country}\SpecialCharTok{==}\StringTok{"Global"}\NormalTok{) }\SpecialCharTok{\%\textgreater{}\%} 
  \FunctionTok{fill}\NormalTok{(LandOceanAvgTemp, }\AttributeTok{.direction =} \StringTok{"down"}\NormalTok{) }\SpecialCharTok{\%\textgreater{}\%}
  \FunctionTok{filter}\NormalTok{(}\SpecialCharTok{!}\FunctionTok{is.na}\NormalTok{(LandOceanAvgTemp)) }\SpecialCharTok{\%\textgreater{}\%} 
  \FunctionTok{model}\NormalTok{(}\AttributeTok{stlf =} \FunctionTok{decomposition\_model}\NormalTok{(}
    \FunctionTok{STL}\NormalTok{(LandOceanAvgTemp }\SpecialCharTok{\textasciitilde{}} \FunctionTok{trend}\NormalTok{(}\AttributeTok{window =} \DecValTok{7}\NormalTok{), }\AttributeTok{robust =} \ConstantTok{TRUE}\NormalTok{),}
        \AttributeTok{Drift =} \FunctionTok{RW}\NormalTok{(season\_adjust }\SpecialCharTok{\textasciitilde{}} \FunctionTok{drift}\NormalTok{())}
\NormalTok{    )) }\SpecialCharTok{\%\textgreater{}\%} 
  \FunctionTok{forecast}\NormalTok{(}\AttributeTok{h =} \StringTok{"8 years"}\NormalTok{) }\SpecialCharTok{\%\textgreater{}\%} 
  \FunctionTok{autoplot}\NormalTok{(}\FunctionTok{filter}\NormalTok{(GlobalCountryTemp, }\FunctionTok{year}\NormalTok{(Date) }\SpecialCharTok{\textgreater{}} \DecValTok{2000}\NormalTok{)) }\SpecialCharTok{+}
  \FunctionTok{labs}\NormalTok{(}\AttributeTok{title =} \StringTok{"Global Land \& Ocean Average Temperature {-} Forecast"}\NormalTok{,}
       \AttributeTok{subtitle =} \StringTok{"Drift method based on seasonal adjusted data"}\NormalTok{,}
       \AttributeTok{y=}\StringTok{"LandOceanAvgTemp Cº"}\NormalTok{,}
       \AttributeTok{caption =} \StringTok{"fig. 37"}\NormalTok{)}
\end{Highlighting}
\end{Shaded}

\includegraphics{Project_files/figure-latex/unnamed-chunk-21-1.pdf}

The figures above show the forecasts for the next 8 years (from 2016
onward) estimated by the ``Drift Method'' based on the seasonal adjusted
data (based on STL Decomposition) for the times Series ``Global Land
Average temperature'' and ``Global Land \& Ocean Temperature'',
respectively. The prediction intervals are still big, especially for the
forecast of the ``Global Land Average temperature''. When we compare the
prediction interval to those of the ``Seasonal naïve method'' in the
figures before they seem to be even bigger.

In the following chapter we will calculate the forecasting accuracy and
check if our intuition about the best method is true.

\hypertarget{accuracy-performance-evaluation}{%
\subsubsection{Accuracy \& performance
evaluation}\label{accuracy-performance-evaluation}}

In this chapter we will divide our data in training and test sets to
evaluate the forecasting accuracy. For this purpose the time series
cross-validation is used. In this procedure, there are a number of test
sets, each consisting of a single observation. The corresponding
training set consists only of observations that occurred before the
observation that forms the test set. By averaging over the test series
the forecasting accuracy is calculated.

In the following we will use the one-step ahead forecast with an initial
sample size of 3 and the stretch\_tsibble function (=extends a growing
length window with new data) to calculate the accuracy obtained via time
series cross-validation. To increase the performance of the code, only
data since 1990 will be considered.

\hypertarget{global-land-average-temperature-forecast}{%
\paragraph{Global land average temperature
forecast}\label{global-land-average-temperature-forecast}}

\begin{Shaded}
\begin{Highlighting}[]
\NormalTok{GlobalCountryTemp }\SpecialCharTok{\%\textgreater{}\%} 
  \FunctionTok{filter}\NormalTok{(Country}\SpecialCharTok{==}\StringTok{"Global"}\NormalTok{) }\SpecialCharTok{\%\textgreater{}\%} 
  \FunctionTok{filter}\NormalTok{(}\FunctionTok{year}\NormalTok{(Date)}\SpecialCharTok{\textgreater{}=}\DecValTok{1990}\NormalTok{) }\SpecialCharTok{\%\textgreater{}\%}
  \FunctionTok{fill}\NormalTok{(LandAvgTemp, }\AttributeTok{.direction =} \StringTok{"down"}\NormalTok{) }\SpecialCharTok{\%\textgreater{}\%}
  \FunctionTok{filter}\NormalTok{(}\SpecialCharTok{!}\FunctionTok{is.na}\NormalTok{(LandAvgTemp)) }\SpecialCharTok{\%\textgreater{}\%}
  \FunctionTok{stretch\_tsibble}\NormalTok{(}\AttributeTok{.init =} \DecValTok{3}\NormalTok{, }\AttributeTok{.step =} \DecValTok{1}\NormalTok{) }\SpecialCharTok{\%\textgreater{}\%} 
  \FunctionTok{model}\NormalTok{(}\AttributeTok{Seasonal\_naive =} \FunctionTok{SNAIVE}\NormalTok{(LandAvgTemp),}
        \AttributeTok{Drift =} \FunctionTok{RW}\NormalTok{(LandAvgTemp }\SpecialCharTok{\textasciitilde{}} \FunctionTok{drift}\NormalTok{())}
\NormalTok{        ) }\SpecialCharTok{\%\textgreater{}\%} \FunctionTok{forecast}\NormalTok{(}\AttributeTok{h =} \StringTok{"8 years"}\NormalTok{) }\SpecialCharTok{\%\textgreater{}\%} 
  \FunctionTok{accuracy}\NormalTok{(GlobalCountryTemp)}
\end{Highlighting}
\end{Shaded}

\begin{verbatim}
## # A tibble: 2 x 11
##   .model    Country .type     ME   RMSE   MAE     MPE   MAPE   MASE  RMSSE  ACF1
##   <chr>     <chr>   <chr>  <dbl>  <dbl> <dbl>   <dbl>  <dbl>  <dbl>  <dbl> <dbl>
## 1 Drift     Global  Test  -5.77  19.9   9.53  -109.   140.   14.0   17.9   0.983
## 2 Seasonal~ Global  Test   0.113  0.455 0.355    1.06   5.30  0.521  0.409 0.319
\end{verbatim}

\begin{Shaded}
\begin{Highlighting}[]
\NormalTok{GlobalCountryTemp }\SpecialCharTok{\%\textgreater{}\%} 
  \FunctionTok{filter}\NormalTok{(Country}\SpecialCharTok{==}\StringTok{"Global"}\NormalTok{) }\SpecialCharTok{\%\textgreater{}\%} 
  \FunctionTok{filter}\NormalTok{(}\FunctionTok{year}\NormalTok{(Date)}\SpecialCharTok{\textgreater{}=}\DecValTok{1990}\NormalTok{) }\SpecialCharTok{\%\textgreater{}\%}
  \FunctionTok{fill}\NormalTok{(LandAvgTemp, }\AttributeTok{.direction =} \StringTok{"down"}\NormalTok{) }\SpecialCharTok{\%\textgreater{}\%}
  \FunctionTok{filter}\NormalTok{(}\SpecialCharTok{!}\FunctionTok{is.na}\NormalTok{(LandAvgTemp)) }\SpecialCharTok{\%\textgreater{}\%}
  \FunctionTok{stretch\_tsibble}\NormalTok{(}\AttributeTok{.init =} \DecValTok{3}\NormalTok{, }\AttributeTok{.step =} \DecValTok{1}\NormalTok{) }\SpecialCharTok{\%\textgreater{}\%}
  \FunctionTok{model}\NormalTok{(}\AttributeTok{stlf =} \FunctionTok{decomposition\_model}\NormalTok{(}
    \FunctionTok{STL}\NormalTok{(LandAvgTemp }\SpecialCharTok{\textasciitilde{}} \FunctionTok{trend}\NormalTok{(}\AttributeTok{window =} \DecValTok{7}\NormalTok{), }\AttributeTok{robust =} \ConstantTok{TRUE}\NormalTok{),}
        \AttributeTok{Drift =} \FunctionTok{RW}\NormalTok{(season\_adjust }\SpecialCharTok{\textasciitilde{}} \FunctionTok{drift}\NormalTok{())}
\NormalTok{    )) }\SpecialCharTok{\%\textgreater{}\%} \FunctionTok{forecast}\NormalTok{(}\AttributeTok{h =} \StringTok{"8 years"}\NormalTok{) }\SpecialCharTok{\%\textgreater{}\%} 
  \FunctionTok{accuracy}\NormalTok{(GlobalCountryTemp)}
\end{Highlighting}
\end{Shaded}

\begin{verbatim}
## # A tibble: 1 x 11
##   .model Country .type    ME  RMSE   MAE   MPE  MAPE  MASE RMSSE  ACF1
##   <chr>  <chr>   <chr> <dbl> <dbl> <dbl> <dbl> <dbl> <dbl> <dbl> <dbl>
## 1 stlf   Global  Test  -3.35  18.2  3.89 -51.0  57.5  5.71  16.4 0.988
\end{verbatim}

In the figures above we can see the accuracy of the three different
models for the Average global Land Temperature. The RMSE is a good way
to choose the best forecasting model. The RMSE of the ``Seasonal naïve
method'' is the smallest, followed by the ``Drift Method'' on seasonal
adjusted data and then the normal ``Drift method''.

\hypertarget{global-land-and-ocean-average-temperature-forecast}{%
\paragraph{Global land and ocean average temperature
forecast}\label{global-land-and-ocean-average-temperature-forecast}}

\begin{Shaded}
\begin{Highlighting}[]
\NormalTok{GlobalCountryTemp }\SpecialCharTok{\%\textgreater{}\%} 
  \FunctionTok{filter}\NormalTok{(Country}\SpecialCharTok{==}\StringTok{"Global"}\NormalTok{) }\SpecialCharTok{\%\textgreater{}\%} 
  \FunctionTok{filter}\NormalTok{(}\FunctionTok{year}\NormalTok{(Date)}\SpecialCharTok{\textgreater{}=}\DecValTok{1990}\NormalTok{) }\SpecialCharTok{\%\textgreater{}\%}
  \FunctionTok{fill}\NormalTok{(LandOceanAvgTemp, }\AttributeTok{.direction =} \StringTok{"down"}\NormalTok{) }\SpecialCharTok{\%\textgreater{}\%}
  \FunctionTok{filter}\NormalTok{(}\SpecialCharTok{!}\FunctionTok{is.na}\NormalTok{(LandOceanAvgTemp)) }\SpecialCharTok{\%\textgreater{}\%}
  \FunctionTok{stretch\_tsibble}\NormalTok{(}\AttributeTok{.init =} \DecValTok{3}\NormalTok{, }\AttributeTok{.step =} \DecValTok{1}\NormalTok{) }\SpecialCharTok{\%\textgreater{}\%} 
  \FunctionTok{model}\NormalTok{(}\AttributeTok{Seasonal\_naive =} \FunctionTok{SNAIVE}\NormalTok{(LandOceanAvgTemp),}
        \AttributeTok{Drift =} \FunctionTok{RW}\NormalTok{(LandOceanAvgTemp }\SpecialCharTok{\textasciitilde{}} \FunctionTok{drift}\NormalTok{())}
\NormalTok{        ) }\SpecialCharTok{\%\textgreater{}\%} \FunctionTok{forecast}\NormalTok{(}\AttributeTok{h =} \StringTok{"8 years"}\NormalTok{) }\SpecialCharTok{\%\textgreater{}\%} 
  \FunctionTok{accuracy}\NormalTok{(GlobalCountryTemp)}
\end{Highlighting}
\end{Shaded}

\begin{verbatim}
## Warning: The future dataset is incomplete, incomplete out-of-sample data will be treated as missing. 
## 96 observations are missing between 2016 jan and 2023 dez
\end{verbatim}

\begin{verbatim}
## # A tibble: 2 x 11
##   .model       Country .type      ME  RMSE   MAE     MPE  MAPE  MASE RMSSE  ACF1
##   <chr>        <chr>   <chr>   <dbl> <dbl> <dbl>   <dbl> <dbl> <dbl> <dbl> <dbl>
## 1 Drift        Global  Test  -1.66   5.88  2.79  -11.2   18.0  18.4  30.1  0.983
## 2 Seasonal_na~ Global  Test   0.0701 0.197 0.156   0.436  1.00  1.03  1.01 0.605
\end{verbatim}

\begin{Shaded}
\begin{Highlighting}[]
\NormalTok{GlobalCountryTemp }\SpecialCharTok{\%\textgreater{}\%} 
  \FunctionTok{filter}\NormalTok{(Country}\SpecialCharTok{==}\StringTok{"Global"}\NormalTok{) }\SpecialCharTok{\%\textgreater{}\%} 
  \FunctionTok{filter}\NormalTok{(}\FunctionTok{year}\NormalTok{(Date)}\SpecialCharTok{\textgreater{}=}\DecValTok{1990}\NormalTok{) }\SpecialCharTok{\%\textgreater{}\%}
  \FunctionTok{fill}\NormalTok{(LandOceanAvgTemp, }\AttributeTok{.direction =} \StringTok{"down"}\NormalTok{) }\SpecialCharTok{\%\textgreater{}\%}
  \FunctionTok{filter}\NormalTok{(}\SpecialCharTok{!}\FunctionTok{is.na}\NormalTok{(LandOceanAvgTemp)) }\SpecialCharTok{\%\textgreater{}\%}
  \FunctionTok{stretch\_tsibble}\NormalTok{(}\AttributeTok{.init =} \DecValTok{3}\NormalTok{, }\AttributeTok{.step =} \DecValTok{1}\NormalTok{) }\SpecialCharTok{\%\textgreater{}\%}
  \FunctionTok{model}\NormalTok{(}\AttributeTok{stlf =} \FunctionTok{decomposition\_model}\NormalTok{(}
    \FunctionTok{STL}\NormalTok{(LandOceanAvgTemp }\SpecialCharTok{\textasciitilde{}} \FunctionTok{trend}\NormalTok{(}\AttributeTok{window =} \DecValTok{7}\NormalTok{), }\AttributeTok{robust =} \ConstantTok{TRUE}\NormalTok{),}
        \AttributeTok{Drift =} \FunctionTok{RW}\NormalTok{(season\_adjust }\SpecialCharTok{\textasciitilde{}} \FunctionTok{drift}\NormalTok{())}
\NormalTok{    )) }\SpecialCharTok{\%\textgreater{}\%} \FunctionTok{forecast}\NormalTok{(}\AttributeTok{h =} \StringTok{"8 years"}\NormalTok{) }\SpecialCharTok{\%\textgreater{}\%} 
  \FunctionTok{accuracy}\NormalTok{(GlobalCountryTemp)}
\end{Highlighting}
\end{Shaded}

\begin{verbatim}
## # A tibble: 1 x 11
##   .model Country .type     ME  RMSE   MAE   MPE  MAPE  MASE RMSSE  ACF1
##   <chr>  <chr>   <chr>  <dbl> <dbl> <dbl> <dbl> <dbl> <dbl> <dbl> <dbl>
## 1 stlf   Global  Test  -0.952  5.40  1.20 -6.20  7.74  7.86  27.6 0.988
\end{verbatim}

In the figures above we can see the accuracy of the three different
models for the Average global Land \& Ocean Temperature. By evaluating
the RMSE we can again see that the forecasting accuracy of the
``Seasonal naïve method'' is the best, followed by the ``Drift Method''
on seasonal adjusted data and then the normal ``Drift method''.

The reason for this may be that our time series shows a strong
seasonality but only a slight trend. Therefore, the ``Seasonal naïve
method'' already captures the most important factor. If the forecasting
horizon would be bigger it can be assumed, that the performance of the
``Drift Method'' on seasonal adjusted data increases. Moreover, a
different value for the step ahead forecast (we chose one-step ahead
forecast) could lead to a different result.

\hypertarget{checking-residuals-of-selected-method}{%
\paragraph{Checking Residuals of selected
method}\label{checking-residuals-of-selected-method}}

Based on our previous analysis, we can conclude that the ``Seasonal
naïve method'' is the most appropriate method for our time series. Let's
have a look at the residuals of this method:

\begin{Shaded}
\begin{Highlighting}[]
\NormalTok{GlobalCountryTemp }\SpecialCharTok{\%\textgreater{}\%} 
  \FunctionTok{filter}\NormalTok{(Country}\SpecialCharTok{==}\StringTok{"Global"}\NormalTok{) }\SpecialCharTok{\%\textgreater{}\%} 
  \FunctionTok{filter}\NormalTok{(}\FunctionTok{year}\NormalTok{(Date)}\SpecialCharTok{\textgreater{}=}\DecValTok{1990}\NormalTok{) }\SpecialCharTok{\%\textgreater{}\%}
  \FunctionTok{fill}\NormalTok{(LandAvgTemp, }\AttributeTok{.direction =} \StringTok{"down"}\NormalTok{) }\SpecialCharTok{\%\textgreater{}\%} 
  \FunctionTok{model}\NormalTok{(}\AttributeTok{Seasonal\_naive =} \FunctionTok{SNAIVE}\NormalTok{(LandAvgTemp)) }\SpecialCharTok{\%\textgreater{}\%} 
  \FunctionTok{augment}\NormalTok{() }\SpecialCharTok{\%\textgreater{}\%} 
  \FunctionTok{autoplot}\NormalTok{(.resid) }\SpecialCharTok{+} \FunctionTok{labs}\NormalTok{(}\AttributeTok{title =} \StringTok{"Residuals from Seasonal naïve method"}\NormalTok{, }\AttributeTok{caption =} \StringTok{"fig. 24"}\NormalTok{)}
\end{Highlighting}
\end{Shaded}

\includegraphics{Project_files/figure-latex/unnamed-chunk-26-1.pdf}

\begin{Shaded}
\begin{Highlighting}[]
\NormalTok{GlobalCountryTemp }\SpecialCharTok{\%\textgreater{}\%} 
  \FunctionTok{filter}\NormalTok{(Country}\SpecialCharTok{==}\StringTok{"Global"}\NormalTok{) }\SpecialCharTok{\%\textgreater{}\%}
  \FunctionTok{filter}\NormalTok{(}\FunctionTok{year}\NormalTok{(Date)}\SpecialCharTok{\textgreater{}=}\DecValTok{1990}\NormalTok{) }\SpecialCharTok{\%\textgreater{}\%}
  \FunctionTok{fill}\NormalTok{(LandAvgTemp, }\AttributeTok{.direction =} \StringTok{"down"}\NormalTok{) }\SpecialCharTok{\%\textgreater{}\%} 
  \FunctionTok{model}\NormalTok{(}\AttributeTok{Seasonal\_naive =} \FunctionTok{SNAIVE}\NormalTok{(LandAvgTemp)) }\SpecialCharTok{\%\textgreater{}\%} 
  \FunctionTok{augment}\NormalTok{() }\SpecialCharTok{\%\textgreater{}\%} 
  \FunctionTok{ACF}\NormalTok{(.resid) }\SpecialCharTok{\%\textgreater{}\%} 
  \FunctionTok{autoplot}\NormalTok{() }\SpecialCharTok{+} \FunctionTok{labs}\NormalTok{(}\AttributeTok{title =} \StringTok{"ACF of Residuals from Seasonal naïve method"}\NormalTok{, }\AttributeTok{caption =} \StringTok{"fig. 38"}\NormalTok{)}
\end{Highlighting}
\end{Shaded}

\includegraphics{Project_files/figure-latex/unnamed-chunk-26-2.pdf}

We have limited our time series to data from 1990 onwards, as we do not
need data from 1850 onwards. In the two figures above we can see the
residuals from the ``Seasonal naïve method'' and the ACF of the
residuals from the ``Seasonal naïve method''. The ACF value is for some
lags very low, but we can also see that there are several outliers.
Therefore, we can't say that the residuals of our selected method looks
like white noise. This means that none of the Benchmark method is
optimal for our time series. However, the ``Seasonal naïve method''
seems to perform good with regard to the forecasting accuracy.

\hypertarget{exponential-smoothing}{%
\subsection{5. Exponential Smoothing}\label{exponential-smoothing}}

The second to last topic in this report will focus entirely in applying
Exponential Smoothing methods to the given time-series. Firstly, ETS
forecasting will be applied to the global average temperatures, as well
as land average temperatures for each continent. Secondly, an assessment
of the forecast accuracy will be analyzed.

\hypertarget{global-land-average-temperature}{%
\subsubsection{Global Land Average
Temperature}\label{global-land-average-temperature}}

Due to the strong presence of seasonality in our data, the most
appropriate Exponential Smoothing methods would most likely be the ones
with a seasonal component, such as Holt-Winters Additive Method or
Holt-Winters Multiplicative Method. Besides that, a more simple ETS
Method without a trend component could also be considered.Therefore, in
this first step, these three exponential smoothing methods are applied.

\begin{Shaded}
\begin{Highlighting}[]
\NormalTok{global\_temp }\OtherTok{\textless{}{-}} 
\NormalTok{  GlobalCountryTemp }\SpecialCharTok{\%\textgreater{}\%} 
  \FunctionTok{filter}\NormalTok{(Country }\SpecialCharTok{==} \StringTok{"Global"}\NormalTok{) }\SpecialCharTok{\%\textgreater{}\%} 
  \FunctionTok{filter}\NormalTok{(}\FunctionTok{year}\NormalTok{(Date)}\SpecialCharTok{\textgreater{}=}\DecValTok{1990}\NormalTok{)}

\NormalTok{fit\_global\_land }\OtherTok{\textless{}{-}}\NormalTok{ global\_temp }\SpecialCharTok{\%\textgreater{}\%} 
  \FunctionTok{model}\NormalTok{(}
    \AttributeTok{ANA\_Land =} \FunctionTok{ETS}\NormalTok{(LandAvgTemp }\SpecialCharTok{\textasciitilde{}} \FunctionTok{error}\NormalTok{(}\StringTok{"A"}\NormalTok{) }\SpecialCharTok{+} \FunctionTok{trend}\NormalTok{(}\StringTok{"N"}\NormalTok{) }\SpecialCharTok{+} \FunctionTok{season}\NormalTok{(}\StringTok{"A"}\NormalTok{)),}
    \AttributeTok{AAA\_Land =} \FunctionTok{ETS}\NormalTok{(LandAvgTemp }\SpecialCharTok{\textasciitilde{}} \FunctionTok{error}\NormalTok{(}\StringTok{"A"}\NormalTok{) }\SpecialCharTok{+} \FunctionTok{trend}\NormalTok{(}\StringTok{"A"}\NormalTok{) }\SpecialCharTok{+} \FunctionTok{season}\NormalTok{(}\StringTok{"A"}\NormalTok{)),}
    \AttributeTok{MAM\_Land =} \FunctionTok{ETS}\NormalTok{(LandAvgTemp }\SpecialCharTok{\textasciitilde{}} \FunctionTok{error}\NormalTok{(}\StringTok{"M"}\NormalTok{) }\SpecialCharTok{+} \FunctionTok{trend}\NormalTok{(}\StringTok{"A"}\NormalTok{) }\SpecialCharTok{+} \FunctionTok{season}\NormalTok{(}\StringTok{"M"}\NormalTok{))}
\NormalTok{    )}

\NormalTok{fit\_ANA }\OtherTok{\textless{}{-}}\NormalTok{ global\_temp }\SpecialCharTok{\%\textgreater{}\%} \FunctionTok{model}\NormalTok{(}\AttributeTok{ANA =} \FunctionTok{ETS}\NormalTok{(LandAvgTemp }\SpecialCharTok{\textasciitilde{}} \FunctionTok{error}\NormalTok{(}\StringTok{"A"}\NormalTok{) }\SpecialCharTok{+} \FunctionTok{trend}\NormalTok{(}\StringTok{"N"}\NormalTok{) }\SpecialCharTok{+} \FunctionTok{season}\NormalTok{(}\StringTok{"A"}\NormalTok{)))}

\FunctionTok{report}\NormalTok{(fit\_global\_land)}
\end{Highlighting}
\end{Shaded}

\begin{verbatim}
## # A tibble: 3 x 10
##   Country .model    sigma2 log_lik   AIC  AICc   BIC    MSE   AMSE    MAE
##   <chr>   <chr>      <dbl>   <dbl> <dbl> <dbl> <dbl>  <dbl>  <dbl>  <dbl>
## 1 Global  ANA_Land 0.0900    -513. 1056. 1058. 1112. 0.0860 0.0947 0.231 
## 2 Global  AAA_Land 0.0909    -514. 1061. 1063. 1125. 0.0862 0.0947 0.231 
## 3 Global  MAM_Land 0.00340   -663. 1360. 1362. 1423. 0.119  0.135  0.0381
\end{verbatim}

\begin{Shaded}
\begin{Highlighting}[]
\NormalTok{fit\_global\_land }\SpecialCharTok{\%\textgreater{}\%} 
  \FunctionTok{components}\NormalTok{() }\SpecialCharTok{\%\textgreater{}\%} 
  \FunctionTok{autoplot}\NormalTok{() }\SpecialCharTok{+} \FunctionTok{labs}\NormalTok{(}\AttributeTok{title =} \StringTok{"ETS Components"}\NormalTok{, }\AttributeTok{caption =} \StringTok{"fig. 39"}\NormalTok{)}
\end{Highlighting}
\end{Shaded}

\includegraphics{Project_files/figure-latex/unnamed-chunk-27-1.pdf} It
seems that when the time series is limited from 1990 onward, the ideal
exponential smoothing method is the one who does not include any trend
(slope). Besides that, the ones that do include a trend (Holt-Winters
Additive and Multiplicative Methods) seem to have a slope that is nearly
equal to zero. The reasoning behind this is most likely related with the
fact that although there is an increasing trend in temperature, this
trend is quite small when we consider a time range from the 1990's
onward. Therefore, the ideal forecasting method in this case, would be
the ``ANA'' ETS, since this is the one that minimizes the AIC, AICc and
BIC.

\begin{Shaded}
\begin{Highlighting}[]
\NormalTok{fit\_global\_land }\SpecialCharTok{\%\textgreater{}\%} 
  \FunctionTok{forecast}\NormalTok{(}\AttributeTok{h =} \StringTok{"8 years"}\NormalTok{) }\SpecialCharTok{\%\textgreater{}\%} 
  \FunctionTok{autoplot}\NormalTok{(}\FunctionTok{filter}\NormalTok{(global\_temp, }\FunctionTok{year}\NormalTok{(Date)}\SpecialCharTok{\textgreater{}}\DecValTok{2000}\NormalTok{),}\AttributeTok{level =} \ConstantTok{NULL}\NormalTok{) }\SpecialCharTok{+} \FunctionTok{labs}\NormalTok{(}\AttributeTok{title =} \StringTok{"Global Land Average Temperature ETS Forecast"}\NormalTok{, }\AttributeTok{caption =} \StringTok{"fig. 40"}\NormalTok{)}
\end{Highlighting}
\end{Shaded}

\includegraphics{Project_files/figure-latex/unnamed-chunk-28-1.pdf} As
expected, the ETS forecasts applied to these time series had a very
small AICc difference and, therefore, even though ANA was the one that
performed the best, the three methods applied do not differ much from
one another. This is mainly due to the fact that the models that did
have a trend component (Holt-Winters Additive and Multiplicative), had a
very small trend value. Naturally, the Holt Winters multiplicative
method was the one that performed the worst because seasonal variations
in this case do not change proportionally to the level of the series.

\begin{Shaded}
\begin{Highlighting}[]
\NormalTok{fit\_ANA }\SpecialCharTok{\%\textgreater{}\%} 
  \FunctionTok{gg\_tsresiduals}\NormalTok{() }\SpecialCharTok{+} \FunctionTok{labs}\NormalTok{(}\AttributeTok{title =} \StringTok{"Residual plots for Global Average Temperature ETS(A,N,A) Forecast"}\NormalTok{, }\AttributeTok{caption =} \StringTok{"fig. 41"}\NormalTok{)}
\end{Highlighting}
\end{Shaded}

\includegraphics{Project_files/figure-latex/unnamed-chunk-29-1.pdf} By
looking at the distribution plot of the residuals, it is possible to
visualize that the residuals do tend to cluster around zero. Besides
that, only 2 out of 24 lags in the ACF are barely significant. This
means that the residuals do seem to be white noise. Which leads to the
conclusion that the residuals are normal and uncorrelated. Therefore,
the forecasts should also follow a normal distribution. This indicates
that using an exponential smoothing method that is comprised of an
additive error with a null trend and additive seasonal component would
yield fairly decent forecasts.

\hypertarget{continents-land-average-temperature}{%
\subsubsection{Continents' Land Average
Temperature}\label{continents-land-average-temperature}}

Similarly to what was done for global land average temperature, this
time around, exponential smoothing methods will be applied to forecast
average temperatures for each continent available in the data set that
is being studied.

\begin{Shaded}
\begin{Highlighting}[]
\NormalTok{land\_continents }\OtherTok{\textless{}{-}} 
\NormalTok{  GlobalCountryTemp }\SpecialCharTok{\%\textgreater{}\%} 
  \FunctionTok{filter}\NormalTok{(Country }\SpecialCharTok{\%in\%} \FunctionTok{c}\NormalTok{(}\StringTok{"North America"}\NormalTok{, }\StringTok{"South America"}\NormalTok{, }\StringTok{"Europe"}\NormalTok{, }\StringTok{"Asia"}\NormalTok{, }\StringTok{"Oceania"}\NormalTok{, }\StringTok{"Africa"}\NormalTok{)) }\SpecialCharTok{\%\textgreater{}\%} 
  \FunctionTok{filter}\NormalTok{(}\FunctionTok{year}\NormalTok{(Date)}\SpecialCharTok{\textgreater{}=}\DecValTok{1990}\NormalTok{)}

\NormalTok{fit\_continents }\OtherTok{\textless{}{-}}\NormalTok{ land\_continents }\SpecialCharTok{\%\textgreater{}\%} 
  \FunctionTok{model}\NormalTok{(}
    \AttributeTok{ANA\_Continents =} \FunctionTok{ETS}\NormalTok{(LandAvgTemp)}
\NormalTok{  )}

\NormalTok{fit\_continents }\SpecialCharTok{\%\textgreater{}\%} \FunctionTok{report}\NormalTok{()}
\end{Highlighting}
\end{Shaded}

\begin{verbatim}
## # A tibble: 6 x 10
##   Country       .model       sigma2 log_lik   AIC  AICc   BIC    MSE  AMSE   MAE
##   <chr>         <chr>         <dbl>   <dbl> <dbl> <dbl> <dbl>  <dbl> <dbl> <dbl>
## 1 Africa        ANA_Contine~  0.105   -477.  984.  986. 1039. 0.0997 0.118 0.242
## 2 Asia          ANA_Contine~  0.618   -730. 1489. 1491. 1544. 0.588  0.597 0.538
## 3 Europe        ANA_Contine~  1.46    -853. 1735. 1737. 1790. 1.39   1.39  0.865
## 4 North America ANA_Contine~  0.662   -739. 1509. 1511. 1564. 0.629  0.680 0.612
## 5 Oceania       ANA_Contine~  0.450   -685. 1399. 1401. 1454. 0.428  0.466 0.497
## 6 South America ANA_Contine~  0.125   -502. 1035. 1037. 1090. 0.119  0.129 0.250
\end{verbatim}

\begin{Shaded}
\begin{Highlighting}[]
\NormalTok{fit\_continents }\SpecialCharTok{\%\textgreater{}\%} 
  \FunctionTok{components}\NormalTok{() }\SpecialCharTok{\%\textgreater{}\%} 
  \FunctionTok{autoplot}\NormalTok{() }\SpecialCharTok{+} \FunctionTok{labs}\NormalTok{(}\AttributeTok{title =} \StringTok{"ETS Components"}\NormalTok{, }\AttributeTok{caption =} \StringTok{"fig. 42"}\NormalTok{)}
\end{Highlighting}
\end{Shaded}

\includegraphics{Project_files/figure-latex/unnamed-chunk-30-1.pdf} The
exponential smoothing method that minimizes the AIC, Corrected AIC and
BIC for all the six continents available in the data set is the ANA
Exponential Smoothing Method. This was also the best exponential
smoothing method for the global land average temperature time series
that we have previously seen.

It is important to highlight that the level and seasonality ETS
components differ drastically from continent to continent. Africa, South
America and Oceania have a level above 20. Whereas Europe, Asia and
North America have a level below 10. This is obviously related with the
fact that temperatures in the latter three continents are significantly
lower than in the former three. Besides that, there seems to be a much
higher seasonal variation in Europe, Asia and North America than in
Africa, Oceania and South America.

\begin{Shaded}
\begin{Highlighting}[]
\NormalTok{fit\_continents }\SpecialCharTok{\%\textgreater{}\%} 
  \FunctionTok{forecast}\NormalTok{(}\AttributeTok{h =} \StringTok{"8 years"}\NormalTok{) }\SpecialCharTok{\%\textgreater{}\%} 
  \FunctionTok{autoplot}\NormalTok{(land\_continents) }\SpecialCharTok{+} \FunctionTok{facet\_grid}\NormalTok{(Country}\SpecialCharTok{\textasciitilde{}}\NormalTok{., }\AttributeTok{scales =} \StringTok{"free\_y"}\NormalTok{, }\AttributeTok{shrink =}\NormalTok{ T)  }\SpecialCharTok{+} \FunctionTok{labs}\NormalTok{(}\AttributeTok{title =} \StringTok{"Land Average Temperature ETS Forecast by Continent"}\NormalTok{, }\AttributeTok{caption =} \StringTok{"fig. 43"}\NormalTok{)}
\end{Highlighting}
\end{Shaded}

\includegraphics{Project_files/figure-latex/unnamed-chunk-31-1.pdf} The
forecasts applied to each time series for each continent clearly differ
significantly from one-another. The seasonal differences that were
previously analyzed are also visible in the ETS forecast since only
North America, Europe and Asia have forecasts of negative or close to
negative temperatures. Additionally, it is important to highlight that
Africa and South America seem to have a larger confidence interval in
its forecast than other continents. Hence, the forecast for these
continents seem to be less precise than for the others.

Lastly, the residuals for the forecast of each continent must be
plotted.

\begin{Shaded}
\begin{Highlighting}[]
\NormalTok{fit\_continents }\SpecialCharTok{\%\textgreater{}\%} 
  \FunctionTok{augment}\NormalTok{() }\SpecialCharTok{\%\textgreater{}\%} 
  \FunctionTok{autoplot}\NormalTok{(.resid) }\SpecialCharTok{+} \FunctionTok{facet\_grid}\NormalTok{(Country}\SpecialCharTok{\textasciitilde{}}\NormalTok{., }\AttributeTok{scales =} \StringTok{"free\_y"}\NormalTok{, }\AttributeTok{shrink =}\NormalTok{ T) }\SpecialCharTok{+} \FunctionTok{labs}\NormalTok{(}\AttributeTok{title =} \StringTok{"Residuals from ANA ETS per Continent"}\NormalTok{, }\AttributeTok{caption =} \StringTok{"fig. 44"}\NormalTok{)}
\end{Highlighting}
\end{Shaded}

\includegraphics{Project_files/figure-latex/unnamed-chunk-32-1.pdf}

\begin{Shaded}
\begin{Highlighting}[]
\NormalTok{fit\_continents }\SpecialCharTok{\%\textgreater{}\%} 
  \FunctionTok{augment}\NormalTok{() }\SpecialCharTok{\%\textgreater{}\%} 
  \FunctionTok{ACF}\NormalTok{(.resid) }\SpecialCharTok{\%\textgreater{}\%} 
  \FunctionTok{autoplot}\NormalTok{() }\SpecialCharTok{+} \FunctionTok{facet\_grid}\NormalTok{(Country}\SpecialCharTok{\textasciitilde{}}\NormalTok{., }\AttributeTok{scales =} \StringTok{"free\_y"}\NormalTok{, }\AttributeTok{shrink =}\NormalTok{ T) }\SpecialCharTok{+} \FunctionTok{labs}\NormalTok{(}\AttributeTok{title =} \StringTok{"Residuals from ANA ETS per Continent"}\NormalTok{, }\AttributeTok{caption =} \StringTok{"fig. 44"}\NormalTok{)}
\end{Highlighting}
\end{Shaded}

\includegraphics{Project_files/figure-latex/unnamed-chunk-32-2.pdf} In a
similar fashion to what was seen for the Global Land Average Temperature
forecast residuals, the residuals for the continents also seem to follow
white noise. Only North America, Europe and Asia have two barely
significant lags in a total of 24. On the other hand, Oceania and South
America have 1, whereas Africa has none. Interestingly enough, the
continents that had two significant lags are the ones with a higher
seasonal variation. Regardless, this means that there is statistical
evidence that indicates that these residuals do follow white noise.

ARIMA (part 5)

\begin{Shaded}
\begin{Highlighting}[]
\FunctionTok{library}\NormalTok{(patchwork)}
\end{Highlighting}
\end{Shaded}

\begin{verbatim}
## Warning: package 'patchwork' was built under R version 4.1.3
\end{verbatim}

\begin{Shaded}
\begin{Highlighting}[]
\NormalTok{GlobalCountryTemp }\SpecialCharTok{\%\textgreater{}\%}  \FunctionTok{filter}\NormalTok{(Country}\SpecialCharTok{==}\StringTok{"Global"}\NormalTok{,  }\FunctionTok{year}\NormalTok{(Date) }\SpecialCharTok{\textgreater{}} \DecValTok{1899}\NormalTok{) }\SpecialCharTok{\%\textgreater{}\%} 
\FunctionTok{autoplot}\NormalTok{(LandAvgTemp)}
\end{Highlighting}
\end{Shaded}

\includegraphics{Project_files/figure-latex/unnamed-chunk-33-1.pdf}

\begin{Shaded}
\begin{Highlighting}[]
\NormalTok{GlobalCountryTemp }\SpecialCharTok{\%\textgreater{}\%}  \FunctionTok{filter}\NormalTok{(Country}\SpecialCharTok{==}\StringTok{"Global"}\NormalTok{,  }\FunctionTok{year}\NormalTok{(Date) }\SpecialCharTok{\textgreater{}} \DecValTok{1899}\NormalTok{) }\SpecialCharTok{\%\textgreater{}\%} 
\FunctionTok{autoplot}\NormalTok{(LandOceanAvgTemp)}
\end{Highlighting}
\end{Shaded}

\includegraphics{Project_files/figure-latex/unnamed-chunk-33-2.pdf} We
plotted the time series of interest in view of beginning the process to
fit ARIMA models to the data. Both showed high seasonality and some
moderate positive linear trend.

\begin{Shaded}
\begin{Highlighting}[]
\NormalTok{GlobalCountryTemp }\SpecialCharTok{\%\textgreater{}\%}  \FunctionTok{filter}\NormalTok{(Country}\SpecialCharTok{==}\StringTok{"Global"}\NormalTok{,  }\FunctionTok{year}\NormalTok{(Date) }\SpecialCharTok{\textgreater{}} \DecValTok{1899}\NormalTok{) }\SpecialCharTok{\%\textgreater{}\%} 
\FunctionTok{autoplot}\NormalTok{(LandAvgTemp }\SpecialCharTok{\%\textgreater{}\%} \FunctionTok{difference}\NormalTok{(}\DecValTok{12}\NormalTok{))}
\end{Highlighting}
\end{Shaded}

\includegraphics{Project_files/figure-latex/unnamed-chunk-34-1.pdf}

\begin{Shaded}
\begin{Highlighting}[]
\NormalTok{GlobalCountryTemp }\SpecialCharTok{\%\textgreater{}\%}  \FunctionTok{filter}\NormalTok{(Country}\SpecialCharTok{==}\StringTok{"Global"}\NormalTok{,  }\FunctionTok{year}\NormalTok{(Date) }\SpecialCharTok{\textgreater{}} \DecValTok{1899}\NormalTok{) }\SpecialCharTok{\%\textgreater{}\%} 
\FunctionTok{autoplot}\NormalTok{(LandOceanAvgTemp }\SpecialCharTok{\%\textgreater{}\%} \FunctionTok{difference}\NormalTok{(}\DecValTok{12}\NormalTok{))}
\end{Highlighting}
\end{Shaded}

\includegraphics{Project_files/figure-latex/unnamed-chunk-34-2.pdf}
After differencing to remove the seasonal trend, these looked far more
stationary.

\begin{Shaded}
\begin{Highlighting}[]
\NormalTok{GlobalCountryTemp }\SpecialCharTok{\%\textgreater{}\%} \FunctionTok{filter}\NormalTok{(Country}\SpecialCharTok{==}\StringTok{"Global"}\NormalTok{,  }\FunctionTok{year}\NormalTok{(Date) }\SpecialCharTok{\textgreater{}} \DecValTok{1899}\NormalTok{)  }\SpecialCharTok{\%\textgreater{}\%}\FunctionTok{mutate}\NormalTok{(}\AttributeTok{diff\_lat =}\NormalTok{ (}\FunctionTok{difference}\NormalTok{(LandAvgTemp, }\DecValTok{12}\NormalTok{)))}\SpecialCharTok{\%\textgreater{}\%}
  \FunctionTok{features}\NormalTok{(diff\_lat, unitroot\_pp)}
\end{Highlighting}
\end{Shaded}

\begin{verbatim}
## # A tibble: 1 x 3
##   Country pp_stat pp_pvalue
##   <chr>     <dbl>     <dbl>
## 1 Global    -26.5      0.01
\end{verbatim}

\begin{Shaded}
\begin{Highlighting}[]
\NormalTok{GlobalCountryTemp }\SpecialCharTok{\%\textgreater{}\%} \FunctionTok{filter}\NormalTok{(Country}\SpecialCharTok{==}\StringTok{"Global"}\NormalTok{,  }\FunctionTok{year}\NormalTok{(Date) }\SpecialCharTok{\textgreater{}} \DecValTok{1899}\NormalTok{)}\SpecialCharTok{\%\textgreater{}\%} \FunctionTok{mutate}\NormalTok{(}\AttributeTok{diff\_lat =}\NormalTok{ (}\FunctionTok{difference}\NormalTok{(LandAvgTemp, }\DecValTok{12}\NormalTok{))) }\SpecialCharTok{\%\textgreater{}\%}
  \FunctionTok{features}\NormalTok{(diff\_lat, unitroot\_kpss)}
\end{Highlighting}
\end{Shaded}

\begin{verbatim}
## # A tibble: 1 x 3
##   Country kpss_stat kpss_pvalue
##   <chr>       <dbl>       <dbl>
## 1 Global     0.0448         0.1
\end{verbatim}

\begin{Shaded}
\begin{Highlighting}[]
\NormalTok{GlobalCountryTemp }\SpecialCharTok{\%\textgreater{}\%} \FunctionTok{filter}\NormalTok{(Country}\SpecialCharTok{==}\StringTok{"Global"}\NormalTok{,  }\FunctionTok{year}\NormalTok{(Date) }\SpecialCharTok{\textgreater{}} \DecValTok{1899}\NormalTok{)  }\SpecialCharTok{\%\textgreater{}\%}\FunctionTok{mutate}\NormalTok{(}\AttributeTok{diff\_lat =}\NormalTok{ (}\FunctionTok{difference}\NormalTok{(LandOceanAvgTemp, }\DecValTok{12}\NormalTok{)))}\SpecialCharTok{\%\textgreater{}\%}
  \FunctionTok{features}\NormalTok{(diff\_lat, unitroot\_pp)}
\end{Highlighting}
\end{Shaded}

\begin{verbatim}
## # A tibble: 1 x 3
##   Country pp_stat pp_pvalue
##   <chr>     <dbl>     <dbl>
## 1 Global    -19.0      0.01
\end{verbatim}

\begin{Shaded}
\begin{Highlighting}[]
\NormalTok{GlobalCountryTemp }\SpecialCharTok{\%\textgreater{}\%} \FunctionTok{filter}\NormalTok{(Country}\SpecialCharTok{==}\StringTok{"Global"}\NormalTok{,  }\FunctionTok{year}\NormalTok{(Date) }\SpecialCharTok{\textgreater{}} \DecValTok{1899}\NormalTok{)}\SpecialCharTok{\%\textgreater{}\%} \FunctionTok{mutate}\NormalTok{(}\AttributeTok{diff\_lat =}\NormalTok{ (}\FunctionTok{difference}\NormalTok{(LandOceanAvgTemp, }\DecValTok{12}\NormalTok{))) }\SpecialCharTok{\%\textgreater{}\%}
  \FunctionTok{features}\NormalTok{(diff\_lat, unitroot\_kpss)}
\end{Highlighting}
\end{Shaded}

\begin{verbatim}
## # A tibble: 1 x 3
##   Country kpss_stat kpss_pvalue
##   <chr>       <dbl>       <dbl>
## 1 Global     0.0747         0.1
\end{verbatim}

First we wanted to check for any necessary differencing or box-cox
transformations before moving into ARIMA modelling. For both series, the
variance appeared to be very stable, so we did not feel that there was
any need for a log transformation. We have already mentioned that there
was high seasonality in the data, which is no surprise as it is to do
with weather. To handle this, we first differenced both series at lag
12. The positive linear trend was very weak but still existed in the
data; we had to decide whether it was necessary to difference it out or
not. We used some tests to help us make the decision; the pp and the
kpss tests. For both series, the pp test had a p-value of 0.01,
rejecting H0 of a unit root, and the kpss test had a p-value of 0.1, not
rejecting H0 that there was not a unit root. The tests agreed that an
extra difference to remove unit root was not necessary for either
series.

\begin{Shaded}
\begin{Highlighting}[]
\CommentTok{\#create variables for seasonally differenced data}
\NormalTok{z1 }\OtherTok{\textless{}{-}}\NormalTok{ GlobalCountryTemp }\SpecialCharTok{\%\textgreater{}\%} \FunctionTok{filter}\NormalTok{(Country}\SpecialCharTok{==}\StringTok{"Global"}\NormalTok{,  }\FunctionTok{year}\NormalTok{(Date) }\SpecialCharTok{\textgreater{}} \DecValTok{1899}\NormalTok{) }\SpecialCharTok{\%\textgreater{}\%} \FunctionTok{mutate}\NormalTok{(}\AttributeTok{diff1 =} \FunctionTok{difference}\NormalTok{(LandAvgTemp, }\DecValTok{12}\NormalTok{))}

\NormalTok{z2 }\OtherTok{\textless{}{-}}\NormalTok{ GlobalCountryTemp }\SpecialCharTok{\%\textgreater{}\%}  \FunctionTok{filter}\NormalTok{(Country}\SpecialCharTok{==}\StringTok{"Global"}\NormalTok{,  }\FunctionTok{year}\NormalTok{(Date) }\SpecialCharTok{\textgreater{}} \DecValTok{1899}\NormalTok{) }\SpecialCharTok{\%\textgreater{}\%} \FunctionTok{mutate}\NormalTok{(}\AttributeTok{diff2 =} \FunctionTok{difference}\NormalTok{(LandOceanAvgTemp, }\DecValTok{12}\NormalTok{)) }
\end{Highlighting}
\end{Shaded}

The next step was to examine the sample autocorrelation and partial
autocorrelation plots, and use them to try and make inference about
possible arima models to fit. For both series, the sacf plot had a
decaying sine wave and the pacf had a fast decay, with lag 3 being the
last significant value. These both suggested that an ARMA(3,0) model be
a good option to try. However, there were some spikes later on in the
pacf plot around lag 12 which implied that the simple ARMA may have
problems.

\begin{Shaded}
\begin{Highlighting}[]
\NormalTok{z1 }\SpecialCharTok{\%\textgreater{}\%} \FunctionTok{gg\_tsdisplay}\NormalTok{(diff1, }\AttributeTok{plot\_type =} \StringTok{\textquotesingle{}partial\textquotesingle{}}\NormalTok{, }\AttributeTok{lag\_max =} \DecValTok{15}\NormalTok{)}
\end{Highlighting}
\end{Shaded}

\includegraphics{Project_files/figure-latex/unnamed-chunk-37-1.pdf}

\begin{Shaded}
\begin{Highlighting}[]
\NormalTok{z2 }\SpecialCharTok{\%\textgreater{}\%} \FunctionTok{gg\_tsdisplay}\NormalTok{(diff2, }\AttributeTok{plot\_type =} \StringTok{\textquotesingle{}partial\textquotesingle{}}\NormalTok{, }\AttributeTok{lag\_max =} \DecValTok{15}\NormalTok{)}
\end{Highlighting}
\end{Shaded}

\includegraphics{Project_files/figure-latex/unnamed-chunk-37-2.pdf} For
the Land temperature series, we trained the following models; an
ARMA(3,0), ARMA(2,0), ARIMA(1,1) and the optimal model from R given by
the ARIMA built in function. We compared these models using the tidy
function, showing a combined AIC, AICc and BIC comparison, and then just
with the AICc for simplicity. We found that the best model was by far
the SARIMA(4,0,0)(2,0,0) model, with an AICc of 863.7; a far lower score
than that of the ARMA(3,0) which had an AICc of 1451.9.

\begin{Shaded}
\begin{Highlighting}[]
\NormalTok{fitting\_land }\OtherTok{\textless{}{-}}\NormalTok{ z1 }\SpecialCharTok{\%\textgreater{}\%}
  \FunctionTok{model}\NormalTok{(}
        \AttributeTok{ar3 =} \FunctionTok{ARIMA}\NormalTok{(diff1 }\SpecialCharTok{\textasciitilde{}} \FunctionTok{pdq}\NormalTok{(}\DecValTok{3}\NormalTok{, }\DecValTok{0}\NormalTok{, }\DecValTok{0}\NormalTok{) }\SpecialCharTok{+} \FunctionTok{PDQ}\NormalTok{(}\DecValTok{0}\NormalTok{,}\DecValTok{0}\NormalTok{,}\DecValTok{0}\NormalTok{)),}
        \AttributeTok{ar2 =} \FunctionTok{ARIMA}\NormalTok{(diff1 }\SpecialCharTok{\textasciitilde{}} \FunctionTok{pdq}\NormalTok{(}\DecValTok{2}\NormalTok{, }\DecValTok{0}\NormalTok{, }\DecValTok{0}\NormalTok{) }\SpecialCharTok{+} \FunctionTok{PDQ}\NormalTok{(}\DecValTok{0}\NormalTok{,}\DecValTok{0}\NormalTok{,}\DecValTok{0}\NormalTok{)), }
        \AttributeTok{arma11 =} \FunctionTok{ARIMA}\NormalTok{(diff1 }\SpecialCharTok{\textasciitilde{}} \FunctionTok{pdq}\NormalTok{(}\DecValTok{1}\NormalTok{, }\DecValTok{1}\NormalTok{, }\DecValTok{1}\NormalTok{) }\SpecialCharTok{+} \FunctionTok{PDQ}\NormalTok{(}\DecValTok{0}\NormalTok{,}\DecValTok{0}\NormalTok{,}\DecValTok{0}\NormalTok{)), }
        \AttributeTok{auto1 =} \FunctionTok{ARIMA}\NormalTok{(diff1)}
\NormalTok{        )}
\NormalTok{fitting\_land }\SpecialCharTok{\%\textgreater{}\%} \FunctionTok{tidy}\NormalTok{()}
\end{Highlighting}
\end{Shaded}

\begin{verbatim}
## # A tibble: 11 x 7
##    Country .model term  estimate std.error statistic   p.value
##    <chr>   <chr>  <chr>    <dbl>     <dbl>     <dbl>     <dbl>
##  1 Global  ar3    ar1     0.310     0.0269     11.6  1.53e- 29
##  2 Global  ar3    ar2     0.115     0.0280      4.12 3.97e-  5
##  3 Global  ar3    ar3     0.0723    0.0269      2.69 7.26e-  3
##  4 Global  ar2    ar1     0.320     0.0267     12.0  1.14e- 31
##  5 Global  ar2    ar2     0.138     0.0267      5.18 2.51e-  7
##  6 Global  auto1  ar1     0.312     0.0270     11.6  1.12e- 29
##  7 Global  auto1  ar2     0.128     0.0282      4.54 5.99e-  6
##  8 Global  auto1  ar3     0.0470    0.0283      1.66 9.66e-  2
##  9 Global  auto1  ar4     0.0781    0.0271      2.88 4.00e-  3
## 10 Global  auto1  sar1   -0.701     0.0260    -27.0  2.54e-129
## 11 Global  auto1  sar2   -0.302     0.0259    -11.6  6.64e- 30
\end{verbatim}

\begin{Shaded}
\begin{Highlighting}[]
\NormalTok{fitting\_land }\SpecialCharTok{\%\textgreater{}\%} \FunctionTok{glance}\NormalTok{() }\SpecialCharTok{\%\textgreater{}\%} \FunctionTok{arrange}\NormalTok{(AICc)}
\end{Highlighting}
\end{Shaded}

\begin{verbatim}
## # A tibble: 3 x 9
##   Country .model sigma2 log_lik   AIC  AICc   BIC ar_roots   ma_roots 
##   <chr>   <chr>   <dbl>   <dbl> <dbl> <dbl> <dbl> <list>     <list>   
## 1 Global  auto1   0.107   -425.  864.  864.  900. <cpl [28]> <cpl [0]>
## 2 Global  ar3     0.166   -722. 1452. 1452. 1473. <cpl [3]>  <cpl [0]>
## 3 Global  ar2     0.166   -726. 1457. 1457. 1473. <cpl [2]>  <cpl [0]>
\end{verbatim}

For the Land and Ocean series, we obtained similar results when model
selecting. The best ARMA model was again the ARMA(3,0), with an AICc of
-1524.2, however this was greatly bettered by the SARIMA(4,0,0)(2,0,0)
which had an AICc of -2120.3.

\begin{Shaded}
\begin{Highlighting}[]
\NormalTok{fitting\_land\_ocean }\OtherTok{\textless{}{-}}\NormalTok{ z2 }\SpecialCharTok{\%\textgreater{}\%}
  \FunctionTok{model}\NormalTok{(}
        \AttributeTok{ar3 =} \FunctionTok{ARIMA}\NormalTok{(diff2 }\SpecialCharTok{\textasciitilde{}} \FunctionTok{pdq}\NormalTok{(}\DecValTok{3}\NormalTok{, }\DecValTok{0}\NormalTok{, }\DecValTok{0}\NormalTok{) }\SpecialCharTok{+} \FunctionTok{PDQ}\NormalTok{(}\DecValTok{0}\NormalTok{,}\DecValTok{0}\NormalTok{,}\DecValTok{0}\NormalTok{)),}
        \AttributeTok{ar2 =} \FunctionTok{ARIMA}\NormalTok{(diff2 }\SpecialCharTok{\textasciitilde{}} \FunctionTok{pdq}\NormalTok{(}\DecValTok{2}\NormalTok{, }\DecValTok{0}\NormalTok{, }\DecValTok{0}\NormalTok{) }\SpecialCharTok{+} \FunctionTok{PDQ}\NormalTok{(}\DecValTok{0}\NormalTok{,}\DecValTok{0}\NormalTok{,}\DecValTok{0}\NormalTok{)), }
        \AttributeTok{arma11 =} \FunctionTok{ARIMA}\NormalTok{(diff2 }\SpecialCharTok{\textasciitilde{}} \FunctionTok{pdq}\NormalTok{(}\DecValTok{1}\NormalTok{, }\DecValTok{1}\NormalTok{, }\DecValTok{1}\NormalTok{) }\SpecialCharTok{+} \FunctionTok{PDQ}\NormalTok{(}\DecValTok{0}\NormalTok{,}\DecValTok{0}\NormalTok{,}\DecValTok{0}\NormalTok{)), }
        \AttributeTok{auto1 =} \FunctionTok{ARIMA}\NormalTok{(diff2)}
\NormalTok{        )}
\NormalTok{fitting\_land\_ocean }\SpecialCharTok{\%\textgreater{}\%} \FunctionTok{tidy}\NormalTok{()}
\end{Highlighting}
\end{Shaded}

\begin{verbatim}
## # A tibble: 11 x 7
##    Country .model term  estimate std.error statistic   p.value
##    <chr>   <chr>  <chr>    <dbl>     <dbl>     <dbl>     <dbl>
##  1 Global  ar3    ar1     0.466     0.0269     17.4  3.28e- 61
##  2 Global  ar3    ar2     0.194     0.0292      6.64 4.61e- 11
##  3 Global  ar3    ar3     0.0691    0.0269      2.57 1.04e-  2
##  4 Global  ar2    ar1     0.482     0.0262     18.4  1.19e- 67
##  5 Global  ar2    ar2     0.227     0.0263      8.64 1.47e- 17
##  6 Global  auto1  ar1     0.477     0.0269     17.7  2.66e- 63
##  7 Global  auto1  ar2     0.202     0.0298      6.77 1.93e- 11
##  8 Global  auto1  ar3     0.0541    0.0298      1.81 7.03e-  2
##  9 Global  auto1  ar4     0.0752    0.0271      2.77 5.66e-  3
## 10 Global  auto1  sar1   -0.711     0.0260    -27.4  3.41e-132
## 11 Global  auto1  sar2   -0.315     0.0258    -12.2  1.08e- 32
\end{verbatim}

\begin{Shaded}
\begin{Highlighting}[]
\NormalTok{fitting\_land\_ocean }\SpecialCharTok{\%\textgreater{}\%} \FunctionTok{glance}\NormalTok{() }\SpecialCharTok{\%\textgreater{}\%} \FunctionTok{arrange}\NormalTok{(AICc)}
\end{Highlighting}
\end{Shaded}

\begin{verbatim}
## # A tibble: 3 x 9
##   Country .model sigma2 log_lik    AIC   AICc    BIC ar_roots   ma_roots 
##   <chr>   <chr>   <dbl>   <dbl>  <dbl>  <dbl>  <dbl> <list>     <list>   
## 1 Global  auto1  0.0124   1067. -2120. -2120. -2084. <cpl [28]> <cpl [0]>
## 2 Global  ar3    0.0192    766. -1524. -1524. -1503. <cpl [3]>  <cpl [0]>
## 3 Global  ar2    0.0192    763. -1520. -1520. -1504. <cpl [2]>  <cpl [0]>
\end{verbatim}

\begin{Shaded}
\begin{Highlighting}[]
\NormalTok{sarima\_land }\OtherTok{\textless{}{-}}\NormalTok{ z1 }\SpecialCharTok{\%\textgreater{}\%} \FunctionTok{model}\NormalTok{(}\AttributeTok{arima =} \FunctionTok{ARIMA}\NormalTok{(diff1))}
\NormalTok{arima\_land }\OtherTok{\textless{}{-}}\NormalTok{ z1 }\SpecialCharTok{\%\textgreater{}\%}  \FunctionTok{model}\NormalTok{(}\AttributeTok{arima =} \FunctionTok{ARIMA}\NormalTok{(diff1 }\SpecialCharTok{\textasciitilde{}} \FunctionTok{pdq}\NormalTok{(}\DecValTok{3}\NormalTok{, }\DecValTok{0}\NormalTok{, }\DecValTok{0}\NormalTok{) }\SpecialCharTok{+} \FunctionTok{PDQ}\NormalTok{(}\DecValTok{0}\NormalTok{,}\DecValTok{0}\NormalTok{,}\DecValTok{0}\NormalTok{))) }
\NormalTok{sarima\_land }\SpecialCharTok{\%\textgreater{}\%} \FunctionTok{gg\_tsresiduals}\NormalTok{(}\AttributeTok{lag\_max=}\DecValTok{15}\NormalTok{)}
\end{Highlighting}
\end{Shaded}

\includegraphics{Project_files/figure-latex/unnamed-chunk-40-1.pdf}

\begin{Shaded}
\begin{Highlighting}[]
\FunctionTok{augment}\NormalTok{(sarima\_land) }\SpecialCharTok{\%\textgreater{}\%}
  \FunctionTok{features}\NormalTok{(.resid, ljung\_box, }\AttributeTok{lag =} \DecValTok{10}\NormalTok{, }\AttributeTok{dof =} \DecValTok{3}\NormalTok{)}
\end{Highlighting}
\end{Shaded}

\begin{verbatim}
## # A tibble: 1 x 4
##   Country .model lb_stat lb_pvalue
##   <chr>   <chr>    <dbl>     <dbl>
## 1 Global  arima     9.19     0.240
\end{verbatim}

\begin{Shaded}
\begin{Highlighting}[]
\NormalTok{arima\_land }\SpecialCharTok{\%\textgreater{}\%} \FunctionTok{gg\_tsresiduals}\NormalTok{(}\AttributeTok{lag\_max=}\DecValTok{15}\NormalTok{)}
\end{Highlighting}
\end{Shaded}

\includegraphics{Project_files/figure-latex/unnamed-chunk-40-2.pdf}

\begin{Shaded}
\begin{Highlighting}[]
\FunctionTok{augment}\NormalTok{(arima\_land) }\SpecialCharTok{\%\textgreater{}\%}
  \FunctionTok{features}\NormalTok{(.resid, ljung\_box, }\AttributeTok{lag =} \DecValTok{10}\NormalTok{, }\AttributeTok{dof =} \DecValTok{3}\NormalTok{)}
\end{Highlighting}
\end{Shaded}

\begin{verbatim}
## # A tibble: 1 x 4
##   Country .model lb_stat lb_pvalue
##   <chr>   <chr>    <dbl>     <dbl>
## 1 Global  arima     4.99     0.661
\end{verbatim}

\begin{Shaded}
\begin{Highlighting}[]
\NormalTok{sarima\_land\_ocean }\OtherTok{\textless{}{-}}\NormalTok{ z2 }\SpecialCharTok{\%\textgreater{}\%}  \FunctionTok{model}\NormalTok{(}\AttributeTok{arima =} \FunctionTok{ARIMA}\NormalTok{(diff2)) }
\NormalTok{arima\_land\_ocean }\OtherTok{\textless{}{-}}\NormalTok{ z2 }\SpecialCharTok{\%\textgreater{}\%}  \FunctionTok{model}\NormalTok{(}\AttributeTok{arima =} \FunctionTok{ARIMA}\NormalTok{(diff2 }\SpecialCharTok{\textasciitilde{}} \FunctionTok{pdq}\NormalTok{(}\DecValTok{3}\NormalTok{, }\DecValTok{0}\NormalTok{, }\DecValTok{0}\NormalTok{) }\SpecialCharTok{+} \FunctionTok{PDQ}\NormalTok{(}\DecValTok{0}\NormalTok{,}\DecValTok{0}\NormalTok{,}\DecValTok{0}\NormalTok{))) }


\NormalTok{sarima\_land\_ocean }\SpecialCharTok{\%\textgreater{}\%} \FunctionTok{gg\_tsresiduals}\NormalTok{(}\AttributeTok{lag\_max=}\DecValTok{15}\NormalTok{)}
\end{Highlighting}
\end{Shaded}

\includegraphics{Project_files/figure-latex/unnamed-chunk-41-1.pdf}

\begin{Shaded}
\begin{Highlighting}[]
\FunctionTok{augment}\NormalTok{(sarima\_land\_ocean) }\SpecialCharTok{\%\textgreater{}\%}
  \FunctionTok{features}\NormalTok{(.resid, ljung\_box, }\AttributeTok{lag =} \DecValTok{10}\NormalTok{, }\AttributeTok{dof =} \DecValTok{3}\NormalTok{)}\CommentTok{\# as there are 3 parameters}
\end{Highlighting}
\end{Shaded}

\begin{verbatim}
## # A tibble: 1 x 4
##   Country .model lb_stat lb_pvalue
##   <chr>   <chr>    <dbl>     <dbl>
## 1 Global  arima     6.39     0.495
\end{verbatim}

\begin{Shaded}
\begin{Highlighting}[]
\NormalTok{arima\_land\_ocean }\SpecialCharTok{\%\textgreater{}\%} \FunctionTok{gg\_tsresiduals}\NormalTok{(}\AttributeTok{lag\_max=}\DecValTok{15}\NormalTok{)}
\end{Highlighting}
\end{Shaded}

\includegraphics{Project_files/figure-latex/unnamed-chunk-41-2.pdf}

\begin{Shaded}
\begin{Highlighting}[]
\FunctionTok{augment}\NormalTok{(arima\_land\_ocean) }\SpecialCharTok{\%\textgreater{}\%}
  \FunctionTok{features}\NormalTok{(.resid, ljung\_box, }\AttributeTok{lag =} \DecValTok{10}\NormalTok{, }\AttributeTok{dof =} \DecValTok{3}\NormalTok{)}
\end{Highlighting}
\end{Shaded}

\begin{verbatim}
## # A tibble: 1 x 4
##   Country .model lb_stat lb_pvalue
##   <chr>   <chr>    <dbl>     <dbl>
## 1 Global  arima     7.73     0.357
\end{verbatim}

We decided to compare the two models further by checking that the
residuals exhibited white noise. For both series we obtained similar
results; the residuals showed white noise behaviour in their plots,
however the autocorrelation function for the residuals of the ARMA(3,0)
showed a strong seasonal component whereas this was far less prominent
in the SARIMA model. The S part of the SARIMA model deals with this
seasonality as expected, further confirming that this should be the
model to use for forecasting.

\begin{Shaded}
\begin{Highlighting}[]
\NormalTok{sarima\_land }\SpecialCharTok{\%\textgreater{}\%} \FunctionTok{forecast}\NormalTok{(}\AttributeTok{h =} \StringTok{"8 years"}\NormalTok{) }\SpecialCharTok{\%\textgreater{}\%} 
  \FunctionTok{autoplot}\NormalTok{(}\FunctionTok{filter}\NormalTok{(z1, }\FunctionTok{year}\NormalTok{(Date)}\SpecialCharTok{\textgreater{}}\DecValTok{2000}\NormalTok{))}
\end{Highlighting}
\end{Shaded}

\includegraphics{Project_files/figure-latex/unnamed-chunk-42-1.pdf}

\begin{Shaded}
\begin{Highlighting}[]
\NormalTok{sarima\_land\_ocean }\SpecialCharTok{\%\textgreater{}\%} \FunctionTok{forecast}\NormalTok{(}\AttributeTok{h =} \StringTok{"8 years"}\NormalTok{) }\SpecialCharTok{\%\textgreater{}\%} 
  \FunctionTok{autoplot}\NormalTok{(}\FunctionTok{filter}\NormalTok{(z2, }\FunctionTok{year}\NormalTok{(Date)}\SpecialCharTok{\textgreater{}}\DecValTok{2000}\NormalTok{))}
\end{Highlighting}
\end{Shaded}

\includegraphics{Project_files/figure-latex/unnamed-chunk-42-2.pdf}

Lastly, to complete the ARIMA box-jenkins process we plotted the
forecast for 8 years in advance for the two series, obtaining what
looked like to be very precise forecasts, especially on the 4 year
margin.

\end{document}
